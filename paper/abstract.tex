Applications of walking robots often call for the ability to walk as
quickly, efficiently, or with as little power as possible.  Gaits to
achieve these objectives may designed manually or learned by repeated
trial and error.
Learning approaches differ in their starting assumptions, some
tweaking the parameters of a hand-tuned model, others exploring a
reasonably compact parameter space, and still others beginning with
few assumptions besides periodicity.

This study compares the performance of two methods of learning gaits:
local search of parameterized motion models and evolution of
artificial neural networks using the HyperNEAT encoding.

We tested six different learning strategies for parameterized gaits,
including uniform and Gaussian random hill climbing, policy gradient
reinforcement learning, Nelder-Mead simplex, and a new method that
uses linear regression to build a model of the fitness landscape and
predict promising areas of parameter space for further exploration.
While all parameter search methods outperformed a manually designed
gait, only the Nelder-Mead simplex and linear regression strategies
beat a random baseline strategy.

The HyperNEAT gaits performed considerably better than all
parameterized local search methods.  Successful evolved gaits showed
complex motion patterns containing multiple fundamental frequencies,
but they also demonstrated reuse of patterns among several motors.
Both served to produce quick gaits.

All tests were performed directly in hardware on a quadruped robot
with nine degrees of freedom.  To the authors' knowledge this is the
first time HyperNEAT gaits have been evolved or tested in hardware.
