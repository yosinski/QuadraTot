\subsection{Learning Methods}

\edit{write this.  Old quadratot Method section below}

%\section{Method}
\seclabel{method}

% Describe in reasonable detail the algorithm you are using to address
% this problem. A pseudo-code description of the algorithm you are
% using is frequently useful. If it makes sense for your project,
% trace through a concrete example, showing how your algorithm
% processes this example. The example should be complex enough to
% illustrate all of the important aspects of the problem but simple
% enough to be easily understood. If possible, an intuitively
% meaningful example is better than one with meaningless symbols.

We use several parameterized motion models that command motors to
positions based on a sine wave, creating a periodic pattern.  While we
investigated several models, for the bulk of our experiments, we used
a model whose five parameters are: amplitude, wavelength, scale inner
vs outer motors, scale left vs right motors, scale back vs front
motors. Each strategy below attempts to choose the best possible
parameters for this motion model.  

We implemented and tested 8 different learning strategies.  All
strategies except for the HyperNEAT method\cite{clune} were
constrained to pick parameters from within predetermined ranges.

\begin{itemize}

\item \emph{Random}: This method randomly generates parameter vectors
  in the allowable range. This strategy is used only as baseline.

\item \emph{Uniform random hill climbing}: This method begins by
  selecting a single random parameter vector.  Subsequent iterations
  generate a neighbor by randomly choosing one parameter to adjust and
  replacing it with a new value chosen with uniform probability in the
  allowable range for that parameter. The neighbor is evaluated by
  running the robot with the newly chosen parameters. If this neighbor
  results in a longer distance walked than the previous best gait, it
  is saved as the new best gait. The process is then repeated, always
  starting with the best gait.

\item \emph{Gaussian random hill climbing}: This method works
  similarly to Uniform random hill climbing, except neighbors are
  generated by adding random Gaussian noise to the current best gait.
  This results in all parameters being changed at once, but the
  resulting vector is always fairly close to the previous best gait.
  We used independently selected noise in each dimension, scaled such
  that the standard deviation of the noise was 5\% of the range of
  that dimension.

\item \emph{N-dimensional policy gradient descent}: As opposed to the
  previous methods, this method explicitly estimates the gradient for
  the objective function. It does this by first evaluating \emph{t}
  randomly generated parameter vectors near the initial vector, each
  dimension of these vectors being perturbed by either $-\epsilon$,
  $0$, or $\epsilon$. Then, for each dimension, it groups vectors into
  three groups: $-\epsilon$, $0$, and $\epsilon$.  The gradient along
  this dimension is then estimated as the average score for the
  $\epsilon$ group minus the average score for the $-\epsilon$
  group. Finally, the method creates a new vector by changing all
  parameters by a fixed-size step in the direction of the gradient.

\item \emph{Nelder-Mead simplex method}\cite{nm}: The Nelder-Mead
  simplex method creates an initial simplex with 6 vertices. The
  initial parameter vector is stored as the first vertex and the other
  five vertices are created by adding to one dimension at a time one
  tenth of the allowable range for that parameter. It then tests the
  fitness of each vertex and based on these fitnesses, it reflects the
  worst point over the centroid in an attempt to improve it.  However,
  to prevent cycles and becoming stuck in local minima, several other
  rules are used.  In general, the worst vertex is reflected over the
  centroid. If the reflected point is better than the second worst
  point and worse than the best point, then the reflected point
  replaces the worst. If the reflected point is better than the best
  point, the simplex is expanded in the direction of the reflected
  point. The better of the reflected and the expanded point replaces
  the worst point. If the reflected point is worse than the second
  worst point, then the simplex is contracted away from the reflected
  point. If the contracted point is better than the reflected point,
  the contracted point replaces the worst point. If the contracted
  point is worse than the reflected point, the entire simplex is
  shrunk \cite{nm}.

\item \emph{Linear regression}: To initialize, this method chooses and
  evaluates five random parameter vectors. It then fits a linear model
  from parameter vector to fitness. In a loop, the method chooses and
  evaluates a new parameter vector generated by taking a fixed-size
  step in the direction of the gradient for each parameter, and fits a
  new linear model to all vectors evaluated so far, choosing the model
  to minimize the sum of squared errors.

\item \emph{SVM regression}: Similarly to linear regression, this
  model starts with several random vectors, but this time they are
  chosen in a small neighborhood about some initial random vector.
  These vectors (generally 8) are evaluated, and a support vector
  regression model is fit to the observed fitnesses.  To choose the
  next vector for evaluation, we randomly generate some number
  (typically 100) of vectors in the neighborhood of the best observed
  gait, and select for evaluation the vector with the best predicted
  performance.  We suspected that if we always chose the best
  predicted point out of 100, we may end up progressing along a narrow
  subspace, prohibiting learning of the true local fitness function.
  Put another way, we would always choose exploitation of knowledge
  vs. exploration of the space.  To address this concern, we added a
  parameter dubbed \code{bumpBy} that added noise to the final
  selected point before it was evaluated.

  Such a method naturally has many tunable parameters, and we
  endeavored to select these parameters by tuning the method in
  simulation.  To estimate the performance of the algorithm, we ran it
  against a simulation with a known optimum.  The simulated function
  was in the same five dimensional parameter space, and simply
  returned a fitness determined as the height of a Gaussian with a
  random mean.  The width of the Gaussian in each dimension was 20\%
  of the range of each dimension, and the maximum value at the peak
  was 100.  \figref{svm_sim_results} shows the learning results on
  this simulated model using the ultimately selected SVM parameters.
  Interestingly, a non-zero value of \code{bumpBy} resulted in better
  learning than noise free (exploration free) learning.

  Ultimately, however, the version of SVM tuned for simulation still
  did not show competitive performance on the real robot.  We tried
  tuning some parameters on the real robot, but after some amount of
  tuning, the method still exhibited too little exploration and easily
  became stuck in local minima.

\item \emph{Evolutionary Neural Network (HyperNEAT)\cite{clune}}: We 
  put together an interface between HyperNEAT
  -- an implementation of a method for evolving neural networks -- and
  the robot, requiring a slightly modified strategy interface.
  

  % Preliminary HyperNEAT runs were promising and resulted in several
  % interesting gaits.  

  % Unfortunately, the gaits generated by HyperNEAT
  % also tended to stress the robot more than typical gaits had before,
  % and the servos would often overheat and malfunction, requiring
  % restarts.  We think these issues may be addressed by adding a small
  % layer between the HyperNEAT strategy and the robot that disallows
  % quickly shifting commanded positions, and we hope to be able to test
  % these methods further once this filter is in place.

\end{itemize}

\acmFig{svm_sim_results}{1}{Results for the SVM regression strategy
  in simulation.  This simulation was used to tune the SVM strategy's
  parameters before trying it on the physical robot.  The strategy
  quickly approaches the maximum simulated fitness of 100.}
