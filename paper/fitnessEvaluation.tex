\subsection{Fitness evaluation details}
\seclabel{fitnessEvaluation}

\edit{Put this in there somewhere:}

To be as fair as possible when obtaining the results presented in
\secref{results}, we picked three random points --- $\vec{\theta}_A$,
$\vec{\theta}_B$, and $\vec{\theta}_C$ --- and started all seven
methods at these points...

\editbox{proofread. especially middle paragraph}

%\section{Experimental Evaluation}

The metric for evaluation of the designed gait was speed. To evaluate
a set of parameters, the robot was sent the parameters and instructed
to walk for a certain length of time. For each evaluation, the robot
always started and ended in the same position in order to measure true
displacement and not reward gaits the ended in a lean, since the LED
would have moved a different distance the robot. More efficient
parameters resulted in a faster gait, which translated into a longer
distance walked, measured in pixels, and a better score.

The size of the Wii remote window is approximately 69 x 47 cm. A
concern about this is that if the robot walks further than usual,
it walks outside the viewable range and thus does not count the run.
This possibility could bias the final results reported on the robot.
% probably could be better written

Each of the parameter ??? methods was run on 3 different initial
parameter vectors, in order to fairly compare the algorithms. We allowed 
each run for the parameter ??? methods to
continue until the results plateaued (no improvement for one third of
the policies seen so far). Three runs of HyperNEAT were completed,
each with a different initial seed and run for 20 generations.
