\subsection{Platform details}
\seclabel{platformDetails}



%\seclabel{implement}

% Describe how you implemented your system and how you structured it. 
% This should give an overview of the system, not a detailed 
% documentation of the code. The documentation of the code is part of 
% the code you hand in. You might want to comment on high-level design 
% decisions that you made. Also explain how you obtained your
% data and any pre-processing you did to it.

\acmFig{topdown.png}{1}{A figure of the robot from a top-down perspective,
with servos labeled}

The quadruped robot used in this study was assembled from parts
printed on the Objet Connex 500 3-D Printing System. It weights 1.88
kg with the on-board computer and measures approximately 38
centimeters from leg to opposite leg in the crouch position depicted
in \figref{robot_close.jpg}. The robot is actuated by 9 AX-12+
Dynamixel servos: one inner joint and one outer joint servo in each of
the four legs, and one servo at the center ``hip'' joint.  This final
unique servo allows the two halves of the robot to rotate with respect
to each other. \figref{topdown.png} shows the positions and numerical
designations of all nine servoes.

Computation for gait learning, fitness evaluation, and robot control
is performed on the compact on-board CompuLab Fit-PC2, running Ubuntu
Linux.  All gait generation, learning, and fitness evaluation code,
except HyperNEAT, is written in Python and is available on our website
\cite{quadraWeb}.  HyperNEAT is written in C++.  We made use of the
excellent \code{pydynamixel} library \cite{pydynamixel} for
controlling the servos, available under the GPLv2 license.  The robot
was set up to connect to a wireless network on boot, and thus we were
able to control all experiments via ssh.

%\acmFigg{wiiMote.jpg}{robot_led.jpg}{1}{A Nintendo Wii remote tracks the
%  location of the robot, providing feedback about distance traveled,
%  in pixels, through an infrared LED mounted on top of the robot.}

\acmFigg{wiiMote_crop.jpg}{robot_led_crop.jpg}{1}{A Nintendo Wii remote tracks the
  location of the robot, providing feedback about distance traveled,
  in pixels, through an infrared LED mounted on top of the robot.}


To track the position of the robot and determine gait fitness, we
mounted a Nintendo Wii remote on the ceiling and an infrared LED on
top of the robot (\figref{wiiMote.JPG}).  The Wii remote contains an
IR camera that can track and report the position of any IR sources in
its image frame.  The resolution of the camera was 1024 by 768 pixels,
which produced a resolution of 1.7mm per pixel when mounted at a
height of \edit{???}m.

A separate tracking server, written in Python, ran on the robot and
interfaced with the Wii remote via bluetooth using the CWiid
library\cite{cwiid}.  Our fitness testing code connected to this
server via socket and requested position updates at the beginning and
end of each run (see \secref{fitnessEvaluation for more details}).

%to interface with the remote via bluetooth. A client then connects via
%a socket to the tracking server and requests position updates
%periodically. If the robot walks beyond the viewable range of the Wii
%remote, a prompt informed the user. The only human intervention
%required during the experiment was to move the robot back inside the
%viewable area and resume the run, and to handle any mechnanical
%failures that arose. This did not interrupt the learning process or
%result in the loss of data.

To run a gait on the robot, one simply provides a \emph{gait
  function}, any Python function that accepts a single input --- time
starting at 0 --- and outputs a list of nine commanded positions, one
for each servo.  To safeguard against limb collision with the robot
body, the robot control code cropped the commands to a safe range.
This range was [150, 770] for the inner leg servos, [30, 680] for the
outer leg servos, and [392, 623] for the center hip servo.


%The robot was set up to run when given any giat 

%is run using a given motion model, including, if desired, 
%smooth interpolation over time for the beginning and end of the run. 

%The servos are prevented from being commanded to a point outside their
%normal range (0 - 1023) as well as beyond points where limbs would collide

