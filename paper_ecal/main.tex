%
% Paper for ECAL conference, formatted per alifexi.sty
% JBY 4/6/11
%



\documentclass[letterpaper]{article}
\usepackage{natbib,alifexi}   %% The order is important
\usepackage{graphicx}
\usepackage{fancyhdr}
\usepackage{color}
%\usepackage{ulem}
\usepackage{ownstyles}
%\usepackage{hyperref}

\graphicspath{{../figures/}}



%\title{Generating gaits for physical quadruped robots: evolved neural networks vs. local parameterized search}
\title{Using HyperNEAT vs. Parameter Optimization to Create Quadruped Robot Gaits}
\author {Jason Yosinski$^{1}$,
Diana Hidalgo$^{1}$,
Sarah Nguyen$^{1}$,
Jeff Clune$^{1}$,
Juan Cristobal Zagal$^{2}$,
\and Hod Lipson$^{1}$\\
\mbox{} \\
$^1$ Cornell University, 239 Upson Hall, Ithaca, NY  14853, USA\\
$^2$ University of Chile, Beauchef 850, Santiago 8370448, Chile\\
jy495@cornell.edu}



\begin{document}

\maketitle

\begin{abstract}
Applications of walking robots often call for the ability to walk as
quickly, efficiently, or with as little power as possible.  Gaits to
achieve these objectives may designed manually or learned by repeated
trial and error.
Learning approaches differ in their starting assumptions, some
tweaking the parameters of a hand-tuned model, others exploring a
reasonably compact parameter space, and still others beginning with
few assumptions besides periodicity.

This study compares the performance of two methods of learning gaits:
local search of parameterized motion models and evolution of
artificial neural networks using the HyperNEAT encoding.

We tested six different learning strategies for parameterized gaits,
including uniform and Gaussian random hill climbing, policy gradient
reinforcement learning, Nelder-Mead simplex, and a new method that
uses linear regression to build a model of the fitness landscape and
predict promising areas of parameter space for further exploration.
While all parameter search methods outperformed a manually designed
gait, only the Nelder-Mead simplex and linear regression strategies
beat a random baseline strategy.

The HyperNEAT gaits performed considerably better than all
parameterized local search methods.  Successful evolved gaits showed
complex motion patterns containing multiple fundamental frequencies,
but they also demonstrated reuse of patterns among several motors.
Both served to produce quick gaits.

All tests were performed directly in hardware on a quadruped robot
with nine degrees of freedom.  To the authors' knowledge this is the
first time HyperNEAT gaits have been evolved or tested in hardware.

\end{abstract}

%%%%%%%%%%%%%%%%%%%%%%

% Section 1
\section{Introduction}

What the problem is...

\figp{aracna_black_2}{.75}{Example figure...}


\subsection{Related Work}



Various machine learning techniques have proven to be effective at generating gaits for legged robots. Kohl and Stone presented a policy gradient reinforcement learning approach for generating a fast walk on legged robots\cite{kohl}. We experiment with this method to create a walk for our robot (called Policy Gradient Descent, described below). Others have evolved gaits for legged robots, producing competitive results~\cite{chernova2005evolutionary, hornby2005autonomous, zykov, clune2009evolving, clune2011performance, clune2009hybrid, clune2009sensitivity, tellez2006evolving, valsalam2008modular}. In fact, an evolved gait was used in the first commercially-available version of Sony's AIBO robot~\cite{hornby2005autonomous}. Except for work with HyperNEAT~\cite{clune2009evolving, clune2011performance, clune2009hybrid, clune2009sensitivity}, the previous evolutionary approaches have helped evolution 'see' the regularity of the problem by manually decomposing the task. In other words, experimenters have to choose which legs should be coordinated, or otherwise facilitate the coordination of robotic legs. Part of the motivation of this paper is to see if the space of regularities can be explored automatically, which has previously performed well~\cite{valsalam2008modular}, and to perform a direct comparison on the same robot to HyperNEAT.  


\subsection{Outline of Sections}

The remainder of the paper is organized as follows. In
\secref{problemDefinition} we define more rigorously the problem of
gait learning. \secref{experimentalSetup} describes our experimental
setup, including the hardware we used and our methods for evaluating
fitness of a gait. \secref{gaitGenLearn} discusses the different gait
generation and learning methods we tested, and \secref{results}
presents and discusses performance results.  Finally,
\secref{conclusion} concludes with possible future extensions to this
work.


% Section 2
\section{Problem definition}
% Precisely define the problem you are addressing (i.e. formally specify
% the inputs and outputs). Elaborate on why this is an interesting and
% important problem.

\edit{write this.  Quadratot report section below}

We are testing several different learning methods to design a
parametrized gait for a quadruped robot from the Cornell Computational
Synthesis Lab.

The output each of the learning algorithms is a function of time,
$f(t)$, that outputs a vector of commanded motor positions.  This
function is generated using a parametrized motion model, described in
\secref{implement}.

The robot executes these commands and measures its change in location
using the tracking system described in \secref{implement}.  The input
to the learning algorithms is this measured displacement, which the
algorithms attempt to maximize. This displacement is measured for each
gait over a constant length run, usually 12 seconds.

A comparison and evaluation of the many different methods available
for optimizing the gait of legged robots will be useful for future
work on this challenging multidimensional control problem.



% Section 3
\section{Experimental Setup}

The quadruped robot used was assembled from parts purchased online and
parts printed by the Objet Connex 500 3-D Printing System. The robot
actuation system consists of 5 AX-18+ Dynamixel servos and 4 AX-12+
Dynamixel servos: one inner joint with one AX-18+ servo and one outer
joint with AX-12+servo in each of the four legs, and one AX-18+ servo
at the center. To avoid the formerly reported AX-18+ servos are used
in this robot because of their stronger actuation power than that of
AX-12+. Each servo could be set to a position in the range [0, 1023]
by using PyDynamixel library, corresponding to a physical range [-120$^\circ$,
  +120$^\circ$]. Also, to prevent collisions with the robot body, the control
module filter out the commands to a safe range. This range was [-85$^\circ$,
  +60$^\circ$] for the inner leg servos, [-113$^\circ$, +39$^\circ$] for the outer leg
servos, and [-28$^\circ$, +28$^\circ$] for the center hip servo. In the studies of
this paper, tethered cables powered both the computer and the
servos. It measures approximately 39.5 centimeters from leg to
opposite leg in the crouch position showed in \figref{quadratotWhiteBooties}. 

Our performance metric was the displacement over the evaluation period of 12 seconds
for each. Same as \cite{yosinski2011evolving-robot-gaits}, the displacement was
measured using a Wii remote that was placed on the ceiling. Different
from the original model described in \cite{yosinski2011evolving-robot-gaits}, the quadruped robot was
equipped with a three-infrared-LED cluster on top rather than just
one. The reason for this setup is that when fierce gaits were
executed, the Wii remote loses tracking of the robot position due to
the limited visible angle of a single LED.  These three LEDs were
place tightly together to act as one signal emitter. Each LED was
tilted outwards in order to maximize the visible range. A
separate tracking server ran on the robot PC interacted with the Wii
remote via bluetooth by using the CWiid library.  A corresponding
client communicates with this server via socket connection. Our
fitness evaluation code talks with the Wii remote by using this client
and updates the position from beginning to end during each run. The
metric for evaluating gaits was the Euclidian distance the robot
traveled during a 12-second run on flat surface. Previous work done in
\cite{yosinski2011evolving-robot-gaits} and \cite{clune2009evolving-coordinated-quadruped} suggested that extremely fierce gaits are not viable in
general. These gaits tend to either overburden the servos or flip the
robot. Thus, RL PoWER were tested after carefully cropping out the
extremely fierce gaits each time. As described earlier, each leg has
two joints, inner joint j1 and outer j2. The position of one leg is
determined by the sum of the values of j1 and j2. Extremely fierce
gaits usually have relatively small values of $g$. As shown in \figref{smartCrop}, this cropping method works
by mapping the wild gaits onto the boundary of a quasi-triangular area in
the two-dimensional space of j1 and j2.

\figp{smartCrop}{1}{The red line is the cropping border, $g = j1 +  j2$. As shown in \figref{sixLegPositions}}

\figp{sixLegPositions}{1}{The left two positions are allowed; middle
  two positions are on the red border line from \figref{smartCrop}; right two positions are deemed as "extremely fierce", thus are disallowed.}

\subsection{Platform details}






\edit{figure:robotDiagram make figure to go somewhere here, show drawing with inner/outer, l/r, and f/b motors.}




%\section{System Architecture and Implementation}
\seclabel{implement}

% Describe how you implemented your system and how you structured it. 
% This should give an overview of the system, not a detailed 
% documentation of the code. The documentation of the code is part of 
% the code you hand in. You might want to comment on high-level design 
% decisions that you made. Also explain how you obtained your
% data and any pre-processing you did to it.
\acmFig{robot_close.jpg}{1}{The robot}
\acmFig{topdown.png}{1}{A figure of the robot from a top-down perspective,
with motors labeled}

The quadruped robot has an on-board computer running Linux. The lower
level drivers are in C and the system is implemented in
Python. 

\acmFig{robot.JPG}{1}{The infrared LED mounted on top of the robot, which
provides feedback about distance traveled in conjunction with a Wii
remote}
\acmFig{wiiMote.JPG}{1}{A Wii remote tracks the location of the robot,
providing feedback about distance traveled}

A Wii remote tracks the location of the robot, providing feedback
about distance traveled, through an infrared LED mounted on top of 
the robot. A server is run on the robot and continuously tracks its
position using the CWiid library\cite{cwiid} to interface with the remote
via bluetooth. A client then connects via a socket to the tracking
server and requests position updates periodically. If the robot moves 
beyond the viewable range of the Wii remote, the system pauses and
directs the user to move the robot back within the viewable range 
before running again.

The robot is run using a given motion model, including, if desired, 
smooth interpolation over time for the beginning and end of the run. 
The servos are prevented from being commanded to a point outside their
normal range (0 - 1023) as well as beyond points where limbs would collide.

\subsection{Fitness evaluation details}

\editbox{write this.  Old quadratot Experimental Evaluation section below.}

%\section{Experimental Evaluation}

%\subsection{Methodology}
% What are the criteria you are using to evaluate your method? What
% specific hypotheses does your experiment test? Describe the 
% experimental methodology that you used. What are the dependent and 
% independent variables? For projects in machine learning, what is 
% the training/test data that was used, and why is it realistic or
% interesting? Exactly what performance data did you collect and how 
% are you presenting and analyzing it? Comparisons to competing methods 
% that address the same problem or to variations of your own algorithm 
% are particularly useful.

The metric for evaluation of the designed gait is speed. We stop each 
run after plateauing results (no improvement for one third of the 
policies seen so far). The standard length of a run designates that it
should be stopped after there is no improvement for one half of the policies
seen so far, but since all runs took place on the actual robot, without use
of a simulator, certain time limitations were imposed on the learning process.

We controlled our experiments from a computer that was connected via a
wireless ethernet to the robot. The robot collected data about
distance walked automatically on its own. If it walked outside of the
Wii remote's viewable area, it informed the user, so the only human
intervention required during an experiment was to move the robot back
inside the viewable area and to resume the run, which did not
interrupt the learning process or result in the loss of data.

We evaluated the efficacy of a set of parameters by sending these
parameters to the robot and instructing it to walk for a certain
length of time. The robot always began from the same position and
returned to the starting position at the end of the run in order to
measure true displacement without giving credit for ending in a leaned
position. More efficient parameters resulted in a faster gait, which
translated into a longer distance walked and a better score. After completing
an evaluation, the robot sent the resulting distance walked back to the
host computer and prepared itself for a new set of parameters to evaluate.

Each algorithm was run on 3 different initial parameter vectors on the
physical robot. We decided to evaluate all methods starting at the
same three vector in order to allow for the fair comparison of each
algorithm.  We evaluate each method based on the amount of improvement
seen from the initial parameter vectors, and on the fastest speeds
achieved during runs.

The resulting gaits from our algorithms quickly outperformed the original
hand-coded walk designed for this robot. The fastest walk, for example, was
4 times faster.


% Section 4
\section{Gait Generation and Learning}
\seclabel{gaitGenLearn}
\subsection{Parametrized Motion Models}
\seclabel{motionModel}

\edit{this might no longer flow...fix} There are a plethora of methods
available for the creation of a vector motion generating function
$g(t)$, and we've chosen two for comparison in this study.

The first method is to choose a family of \emph{parametrized
  functions}.  By fixing the parameters to a particular set of values,
we obtain a deterministic motion function over time.  We tried several
parametrizations on the robot and, upon obtaining reasonable early
success, settled on one particular parametrization.  We dubbed this
the \emph{SineModel5}, as it has five parameters and employs a sine
wave as its root pattern.

The five parameters in $\vec{\theta}$ are as follows:

 - amp - Amplitude       - allowable range
 - tau   - sine wave period   - allowable range
 - multIO - multiplier scaling inner joings from outer joints  - allowable range
 - multFB - multiplier scaling front joints from back joints  - allowable range
 - multLR - multiplier scaling left joints from right joints  - allowable range
(verify this)

\newcommand{\amp}{\mathrm{amp}}

Inuitively, SineModel5 starts with 8 identicaly sine waves of
amplitude $\amp$ and period $\tau$, multiplies the waves for all inner
motors by multIO, multiplies the waves for all front motors by multFB,
and multiplies the waves for all left motors by multLR.  To obtain the
actual motor position commands, these waves are offset by an
appropriate fixed constant so that the base position (when the sine
waves are at 0) is approximately a crouch (the position shown in
\figref{crouchingRobot}).  Finally, in this motion model, the ninth
"hip" motor is kept at a neutral position.  Thus, the commanded
position for all motors, as a vector function of time, is:

\begin{verbatim}
        [g_0(t) ]    [ \amp * sin(\tau t) ]
        [g_1(t) ]    [ \amp * sin(\tau t) ]
        [g_2(t) ]    [ \amp * sin(\tau t) ]
g(t)   [g_3(t) ] = [ \amp * sin(\tau t) ]
        [g_4(t) ]    [ \amp * sin(\tau t) ]
        [g_5(t) ]    [ \amp * sin(\tau t) ]
        [g_6(t..) ]    [ \amp * sin(\tau t) ]
\end{verbatim}

\edit{make figure to go here, show drawing with inner/outer, l/r, and f/b motors.}
\edit{JMC: Use photo of physical robot in this image, and add labels in photoshop?}
The nine subscripted functions correspond to the nine labeled motors in \figref{robotDiagram}

% used is a function parametrized by five numbers, shown in \figref{foo}





\subsection{Learning Methods for Parametrized Motion Models}
\seclabel{learningMethods}

Using the SineModel5 parametrized motion model described in the
previous section, along with the allowable ranges for each of the five
parameters (shown in \tabref{parameterTable}), the task becomes how to
choose the combination of five parameters that results in the fastest
motion, per the evaluation methods in \secref{fitnessEvaluation}.

If we choose a value for the five dimensional parameter
$\vec{\theta}$, then a given physical trial gives us one measurement
of the fitness of that parameter vector, or $f(\vec{\theta})$.  Two
things make learnign difficult.  First, each evaluation of
$f(\vec{\theta})$ is expensive, taking 15-20 seconds of time on
average.  Second, the fitness returned by such evaluations has proven
to be very noisy, with the standard deviation of the noise often being
roughly equivalent to the size of the measurement.

For both of these reasons, an intelligent learning algorithm is
needed.

In this context, by learning algorithm we simply mean a method for
choosing:

1) The value for $\vec{\theta}$ for the initial trial, and

2) The next value of $\vec{\theta}$ to try, given the $\vec{\theta}$
and $f(\vec{\theta})$ for all previous trials

We evaluated a total of eight subtly or not so subtly different
learning methods.  All employed a simple random sampling method for
requirement (1); that is, all methods picked their initial
$\vec{\theta}$ value via uniform random sampling within the allowed
parameter ranges. \edit{make sure this is true}.  Thus, the
differences in the algorithms was how the selected new $\vec{\theta}$
values to try from their past experience.

The eight methods used, and their methods for choosing the next
$\vec{\theta}$ are:




\edit{write this.  Old quadratot Method section below alsdkfj }

%\section{Method}
%\seclabel{method}

% Describe in reasonable detail the algorithm you are using to address
% this problem. A pseudo-code description of the algorithm you are
% using is frequently useful. If it makes sense for your project,
% trace through a concrete example, showing how your algorithm
% processes this example. The example should be complex enough to
% illustrate all of the important aspects of the problem but simple
% enough to be easily understood. If possible, an intuitively
% meaningful example is better than one with meaningless symbols.

%We use several parametrized motion models that command motors to
%positions based on a sine wave, creating a periodic pattern.  While we
%investigated several models, for the bulk of our experiments, we used
%a model whose five parameters are: amplitude, wavelength, scale inner
%vs outer motors, scale left vs right motors, scale back vs front
%motors. Each strategy below attempts to choose the best possible
%parameters for this motion model.  

%We implemented and tested 8 different learning strategies.  All
%strategies except for the HyperNEAT method\cite{clune} were
%constrained to pick parameters from within predetermined ranges.

\edit{JMC:Bullets=narrower column=wasted space. subsubsection instead?}

\begin{itemize}

\item \emph{Random}: This method randomly generates parameter vectors
  in the allowable range for every trial. This strategy was used only as baseline for comparison.

\item \emph{Uniform random hill climbing}: This method begins by
  selecting a single random parameter vector.  Subsequent iterations
  generate a neighbor by randomly choosing one parameter to adjust and
  replacing it with a new value chosen with uniform probability in the
  allowable range for that parameter. The neighbor is evaluated by
  running the robot with the newly chosen parameters. If this neighbor
  results in a longer distance walked than the previous best gait, it
  is saved as the new best gait. The process is then repeated, always
  starting with the best gait.

\item \emph{Gaussian random hill climbing}: This method works
  similarly to Uniform random hill climbing, except neighbors are
  generated by adding random Gaussian noise to the current best gait.
  This results in all parameters being changed at once, but the
  resulting vector is always fairly close to the previous best gait.
  We used independently selected noise in each dimension, scaled such
  that the standard deviation of the noise was 5\% of the range of
  that dimension.

\item \emph{N-dimensional policy gradient descent}: In \cite{kohl},
  Kohl and Stone document a method for local gradient ascent for gait
  learning with noisy fitness evaluations, and we have included this
  method in our evaluation.  This strategy explicitly estimates the
  gradient for the objective function. It does this by first
  evaluating \emph{n} randomly generated parameter vectors near the
  initial vector, each dimension of these vectors being perturbed by
  either $-\epsilon$, $0$, or $\epsilon$. Then, for each dimension, it
  groups vectors into three groups: $-\epsilon$, $0$, and $\epsilon$.
  The gradient along this dimension is then estimated as the average
  score for the $\epsilon$ group minus the average score for the
  $-\epsilon$ group. Finally, the method creates a new vector by
  changing all parameters by a fixed-size step in the direction of the
  gradient.

\item \emph{Nelder-Mead simplex method}: The Nelder-Mead simplex
  method \cite{nm} creates an initial simplex with $d+1$ vertices,
  where $d$ is the dimension of the parameter space. The initial
  parameter vector is stored as the first vertex and the other five
  vertices are created by adding to one dimension at a time one tenth
  of the allowable range for that parameter. It then tests the fitness
  of each vertex and based on these fitnesses, it reflects the worst
  point over the centroid in an attempt to improve it.  In general,
  the worst vertex is reflected; however, to prevent cycles and
  becoming stuck in local minima, several other rules are used.  If
  the reflected point is better than the second worst point and worse
  than the best point, then the reflected point replaces the worst. If
  the reflected point is better than the best point, the simplex is
  expanded in the direction of the reflected point. The better of the
  reflected and the expanded point replaces the worst point. If the
  reflected point is worse than the second worst point, then the
  simplex is contracted away from the reflected point. If the
  contracted point is better than the reflected point, the contracted
  point replaces the worst point. If the contracted point is worse
  than the reflected point, the entire simplex is shrunk \cite{nm}.

\item \emph{Linear regression}: To initialize, this method chooses and
  evaluates five random parameter vectors. It then fits a linear model
  from parameter vector to fitness. In a loop, the method chooses and
  evaluates a new parameter vector generated by taking a fixed-size
  step in the direction of the gradient for each parameter, and fits a
  new linear model to all vectors evaluated so far, choosing the model
  to minimize the sum of squared errors.

\item \emph{SVM regression}: Similarly to linear regression, this
  model starts with several random vectors, but this time they are
  chosen in a small neighborhood about some initial random vector.
  These vectors (generally 8) are evaluated, and a support vector
  regression model is fit to the observed fitnesses.  To choose the
  next vector for evaluation, we randomly generate some number
  (typically 100) of vectors in the neighborhood of the best observed
  gait, and select for evaluation the vector with the best predicted
  performance.  We suspected that if we always chose the best
  predicted point out of 100, we may end up progressing along a narrow
  subspace, prohibiting learning of the true local fitness function.
  Put another way, we would always choose exploitation of knowledge
  vs. exploration of the space.  To address this concern, we added a
  parameter dubbed \code{bumpBy} that added noise to the final
  selected point before it was evaluated.

  Such a method naturally has many tunable parameters, and we
  endeavored to select these parameters by tuning the method in
  simulation.  To estimate the performance of the algorithm, we ran it
  against a simulation with a known optimum.  The simulated function
  was in the same five dimensional parameter space, and simply
  returned a fitness determined as the height of a Gaussian with a
  random mean.  The width of the Gaussian in each dimension was 20\%
  of the range of each dimension, and the maximum value at the peak
  was 100.  \figref{svm_sim_results} shows the learning results on
  this simulated model using the ultimately selected SVM parameters.
  Interestingly, a non-zero value of \code{bumpBy} resulted in better
  learning than noise free (exploration free) learning.

  \edit{JMC: This next paragraph sounds like it belongs in Results?}
  Ultimately, however, the version of SVM tuned for simulation still
  did not show competitive performance on the real robot.  We tried
  tuning some parameters on the real robot, but after some amount of
  tuning, the method still exhibited too little exploration and easily
  became stuck in local minima.

\item \emph{Neuroevolution (Evolving Artificial Neural Networks with HyperNEAT)}: 
    
  HyperNEAT is an indirect encoding for evolving artificial neural networks (ANNs) that is inspired by the way natural organisms develop~\cite{stanley2009hypercube}. It evolves Compositional Pattern Producing Networks (CPPNs)~\cite{stanley2007compositional}, each of which is a genome that encodes an ANN phenotype~\cite{stanley2009hypercube}. Each CPPN is itself a directed graph, where each node is a mathematical function, such as sine or Gaussian. The nature of these functions can facilitate the evolution of properties such as symmetry (e.g., an absolute value or Gaussian function) and repetition (e.g., a sine function)~\cite{stanley2009hypercube, stanley2007compositional}. The signal on each link in the CPPN is multiplied by that link's weight, which can alter its effect.
  
A CPPN is queried once for each link in the ANN substrate to determine that link's weight. The inputs to the CPPN are the Cartesian coordinates of both the source (e.g., $x = 2$, $y = 4$) and target (e.g., $x = 3$, $y = 5$) nodes of a link and a constant bias value. The CPPN takes these five values as inputs and produces one or two output values, depending on the substrate topology. If there is no hidden layer in the substrate, the single output is the weight of the link between a source node on the input layer and a target node on the output layer. If there is a hidden layer, the first output value determines the weight of the link between the associated input (source) and hidden layer (target) nodes, and the second output value determines the weight of the link between the associated hidden (source) and output (target) layer nodes. All pairwise combinations of source and target nodes are iteratively passed as inputs to a CPPN to deter-mine the weight of each substrate link. 
HyperNEAT is capable of exploiting the geometry of a problem: because the link values between nodes in the final ANN substrate are a function of the geometric positions of those nodes, HyperNEAT can exploit such information when solving a problem~\cite{stanley2009hypercube, clune2009sensitivity, clune2011performance}. In the case of quadruped locomotion, this property helped Hyper-NEAT produce gaits with front-back, left-right, and four-way symmetries~\cite{clune2009evolving, clune2011performance}.
  
The evolution of the population of CPPNs occurs according to the principles of the NeuroEvolution of Augmenting Topologies (NEAT) algorithm~\cite{stanley2002evolving}, which was originally designed to evolve ANNs. NEAT can be fruitfully applied to CPPNs because of their structural similarity to ANNs. For example, mutations can add a node, and thus a function, to a CPPN graph, or change its link weights. The NEAT algorithm is unique in three main ways~\cite{stanley2002evolving}. Initially, it starts with small genomes that encode simple networks and slowly complexifies them via mutations that add nodes and links to the network, enabling the algorithm to evolve the network topology in addition to its weights. Secondly, NEAT has a fitness sharing mechanism that preserves diversity in the system and gives time for new innovations to be tuned by evolution before competing them against more adapted rivals. Finally, NEAT tracks historical information to perform intelligent crossover while avoiding the need for expensive topological analysis. A full explanation of NEAT can be found in~\cite{stanley2002evolving}]. 
  
\edit{JMC: fix next paragraph}  
Following Clune et al.~\cite{clune2011performance, clune2009evolving}, the ANNs for the robot consist of three 5x4 Cartesian grids of nodes forming input, hidden, and output layers. Adjacent layers were completely connected, meaning that there were (5x4)2 x 2 = 800 links in each substrate. The inputs to the substrate were the current angles of each of the 12 joints of the robot (described below), a touch sen-sor that provides a 1 if the lower leg is touching the ground and a 0 if it is not, the pitch, roll, and yaw of the torso, and a modified sine wave (to facilitate the production of periodic behaviors). The sine wave was the following function of time (t) in milli-seconds: sin(120 x t) x �. This function produces numbers from й?to �, which was the range of the unconstrained joints. During preliminary tests, we experimentally found the constant 120 to produce fast, natural gaits, and determined that the touch, pitch, roll, yaw, and sine inputs all contributed to the ability to evolve fast gaits [3].

The ANN substrate outputs were the desired angles for each joint, which were fed into proportional controllers that applied forces to move the joints toward the desired angles. Robots were evaluated in the ODE physics simulator (www.ode.org). The rectangular torso of the robot was (in ODE units) 0.15 wide, 0.3 long, and .05 tall. Each of four legs was composed of three cylinders (length 0.075, radius 0.02) and three hinge joints. The first cylinder functioned as a hip bone. It was parallel to the proximal-distal axis of the torso and barely stuck out from it. The other two cylinders were the upper and lower leg. There were two hip joints and one knee joint. The first hip joint allowed the legs to swing forward and backward (anterior-posterior) and was constrained to 180� such that, at maximum extension, it was parallel with the torso. The second hip joint allowed a leg to swing in and out (proximal-distal). Together, the two hip joints approximated a universal joint. The knee joint swung forward and backward. The second hip and knee joints were unconstrained.  

Each controller in a population of 150 was simulated for 3000 time steps (3 sec-onds). Experiments lasted 1000 generations with a switch point at generation 500. Trials were cut short if any part of the robot except its lower leg touched the ground, or if the number of direction changes in joints exceeded 960. The latter condition re-flects the fact that servo motors cannot be vibrated incessantly without breaking. The fitness of controllers was the following function of d, the maximum distance traveled:  . The exponential nature of the function magnified the selective advantage of small increases in the distance traveled. Because HyperNEAT greatly outperforms FT-NEAT on this problem [3], we compare HybrID to only HyperNEAT. 


  

  % Preliminary HyperNEAT runs were promising and resulted in several
  % interesting gaits.  

  % Unfortunately, the gaits generated by HyperNEAT
  % also tended to stress the robot more than typical gaits had before,
  % and the servos would often overheat and malfunction, requiring
  % restarts.  We think these issues may be addressed by adding a small
  % layer between the HyperNEAT strategy and the robot that disallows
  % quickly shifting commanded positions, and we hope to be able to test
  % these methods further once this filter is in place.

\end{itemize}

\acmFig{svm_sim_results}{1}{Results for the SVM regression strategy
  in simulation.  This simulation was used to tune the SVM strategy's
  parameters before trying it on the physical robot.  The strategy
  quickly approaches the maximum simulated fitness of 100.}

\subsection{HyperNEAT Motion Model}
\seclabel{hyperNeatMethod}
    
HyperNEAT is an indirect encoding for evolving artificial neural
networks (ANNs) that is inspired by the way natural organisms
develop~\cite{stanley2009hypercube}. It evolves Compositional Pattern
Producing Networks (CPPNs)~\cite{stanley2007compositional}, each of
which is a genome that encodes an ANN
phenotype~\cite{stanley2009hypercube}. Each CPPN is itself a directed
graph, where the nodes in the graph are mathematical functions, such as sine or
Gaussian. The nature of these functions can facilitate the evolution
of properties such as symmetry (e.g.\ a Gaussian function) and repetition (e.g.\ a sine
function)~\cite{stanley2009hypercube, stanley2007compositional}. The
signal on each link in the CPPN is multiplied by that link's weight,
which can magnify or diminish its effect.
  
A CPPN is queried once for each link in the ANN phenotype to determine
that link's weight~(\figref{hyperneatExplanation.png}). The inputs to the CPPN are the Cartesian
coordinates of both the source (e.g.\ $x = 2$, $y = 4$) and target
(e.g.\ $x = 3$, $y = 5$) nodes of a link and a constant bias
value. The CPPN takes these five values as inputs and produces two output values. The first output value
determines the weight of the link between the associated input
(source) and hidden layer (target) nodes, and the second output value
determines the weight of the link between the associated hidden
(source) and output (target) layer nodes. All pairwise combinations of
source and target nodes are iteratively passed as inputs to a CPPN to
determine the weight of each ANN link.

\acmFig{hyperneatExplanation.png}{1}{HyperNEAT Produces ANNs from
  CPPNs. ANN weights are specified as a function of the geometric
  coordinates of the source node and the target node for each
  connection. The coordinates of these nodes and a constant bias are
  iteratively passed to the CPPN to determine each connection
  weight. The CPPN has two output values, which specify the weights
  for each connection layer as shown.}

HyperNEAT is capable of exploiting the geometry of a problem because
the link values between nodes in the final ANN phenotype are a
function of the geometric positions of those nodes~\cite{stanley2009hypercube, clune2009sensitivity,
  clune2011performance}. In the case of quadruped locomotion, this
property has previously been shown to help HyperNEAT produce gaits with front-back, left-right,
and four-way symmetries~\cite{clune2009evolving,
  clune2011performance}.
  
The evolution of the population of CPPNs occurs according to the
principles of the NeuroEvolution of Augmenting Topologies (NEAT)
algorithm~\cite{stanley2002evolving}, which was originally designed to
evolve ANNs. NEAT can be fruitfully applied to CPPNs because of their
structural similarity to ANNs. For example, mutations can add a node,
and thus a function, to a CPPN graph, or change its link weights. The
NEAT algorithm is unique in three main
ways~\cite{stanley2002evolving}. Initially, it starts with small
genomes that encode simple networks and slowly complexifies them via
mutations that add nodes and links to the network, enabling the
algorithm to evolve the topology of an ANN in addition to its
weights. Secondly, NEAT has a fitness-sharing mechanism that preserves
diversity in the system and gives time for new innovations to be tuned
by evolution before competing them against more adapted
rivals. Finally, NEAT tracks historical information to perform
intelligent crossover while avoiding the need for expensive
topological analysis. A full explanation of NEAT can be found
in~\cite{stanley2002evolving}.
  
The ANN configuration follows previous studies that evolved quadruped
gaits with HyperNEAT in simulation~\cite{clune2011performance,
  clune2009evolving}, but was adapted to accommodate the physical robot used in this paper. Specifically, the ANNs consists of three $3 \times 4$
Cartesian grids of nodes forming input, hidden, and output
layers (\figref{SpiderANN.jpg}). Adjacent layers were allowed to be completely connected, meaning that there
could be $(3 \times 4)^2= 288$ links in each ANN (although evolution can set weights to 0, functionally eliminating the connection). The inputs to the
substrate were the current angles of each of the 9 joints of the robot
(described below \edit{Jeff: is it?}) and a sine and cosine wave (to facilitate the
production of periodic behaviors). The sine and cosine waves were the
following function of time ($t$) in milli-seconds:
$\sin(t/12)\pi$ \editbox{Jeff: this can't be right, because it would imply a period of only 12*2*pi=75 ms. Perhaps this was as a function of time in units of timesteps?}. This function produces numbers in the range [$-\pi,
  \pi$] \edit{Jeff: this is actually [-1,1], right?}, which were then JASON FILL ME IN TO DESCRIBE HOW JOINT
COMMANDS WERE SPECIFIED.

\editbox{Jeff: proofread this paragraph to make sure I didn't lie (too
  much)}

The outputs of the substrate at each time step were nine numbers in
the range $[-1,1]$, which were scaled according to the allowable
ranges for each of the nine motors and then used as commanded
positions.  Occasionally HyperNEAT would produce networks that
exhibited rapid oscillatory behavior, switching between nearly -1 and
nearly 1 from one time step to the next.  This resulted in motor
commands to alternate extremes every 25ms (using a command rate of
40Hz), which tended to damage and overheat the motors.  To ameliorate
this problem, we simply requested four times as many commanded
positions from the ANN and averaged over four commands at a time to
obtain the actual $g(t)$ to be used as a gait.  This proved to work
well and did not restrict the expressiveness of HyperNEAT.

%\editbox{are we mentioning that we gave HyperNEAT the center joint
%  here or earlier?}

\acmFig{SpiderANN.jpg}{1}{ANN Configuration for HyperNEAT Runs. The
  first two columns of each row of the input layer receive information
  about a single leg (the current angle of its two joints). The final
  column provides a sine and cosine wave to enable periodic movements
  and the angle of the center joint. Evolution determines the function
  of the hidden-layer nodes. The nodes in the output layer specify new
  joint angles for each respective joint. The unlabeled nodes in the
  input and output layers are ignored.}\label{hyperneatLayout}

As with the parameterized learning methods, three runs of HyperNEAT
were performed. The population size for HyperNEAT was 9 and runs
lasted 20 generations. These numbers are small, but were necessarily
constrained given how much time it took to conduct evolution directly
on a real robot. The remaining parameters were identical to Clune et
al.~\cite{clune2011performance}.
  

% Preliminary HyperNEAT runs were promising and resulted in several
% interesting gaits.

% Unfortunately, the gaits generated by HyperNEAT also tended to
% stress the robot more than typical gaits had before, and the servos
% would often overheat and malfunction, requiring restarts.  We think
% these issues may be addressed by adding a small layer between the
% HyperNEAT strategy and the robot that disallows quickly shifting
% commanded positions, and we hope to be able to test these methods
% further once this filter is in place.


% Section 5
HyperNEAT, used with a physical simulator, produced gaits that outperformed all previous gaits tried and recorded on the QuadraTot. 
The fastest gait evolved by HyperNEAT in simulation and transfered onto the real robot recorded a speed of 14.5 cm/s, faster than the previously fastest gait published in Glette et al. \cite{glette}. 
In their paper, Kyrre et al. \cite{glette} recorded 17.8 cm/s in their robot in Norway \cite{glette}, but to make sure that there were no differences in robots, the same gait was also tested on the robot used for our experimental testing.
The same gait, ran by using the gait file sent to us by Kyrre Glette, recorded an average speed of 12.951 cm/s and a maximum of 13.76 cm/s on the same QuadraTot platform in the Cornell Creative Machines Lab where Yosinski et al. \cite{yos:clune} and this study was done.
HyperNEAT, used with a simulator, outperformed Kyrre et al's \cite{glette} gait by 5.4\% with much fewer evaluations performed per run, at 40000 evaluations per run with 200 organisms and 200 generations as opposed to Kyrre et al's \cite{glette} 60000 evaluations per run with 200 organisms and 300 generations.
The best gait evolved using HyperNEAT with simulator performed 49.5\% better than the best gait evolved using HyperNEAT in hardware \cite{yos:clune}. 
This is most likely due to the difference in the number of evaluations between the two studies. 
The previous study, HyperNEAT in hardware, only had 180 evaluations while in this study, a simulator allowed a much higher number -- 20000 -- of evaluations. 
Another minor reason for this may be that the implementation placed to reduce gait frequency may have helped in producing gaits that were less susceptible to shutdowns, as Yosinski et al. \cite{yos:clune} did not have such things.
Although most gaits started off its evolution process by producing fast and high frequency gaits, with more generations, the gaits dramatically reduced its frequencies while continually increasing their fitness. The average frequency of the final gaits was 1.29 Hz per servo. 

Gaits produced in this study confirmed that with the use of a physical simulator, HyperNEAT is able to fully exploit the geometry of the problem without human intervention, which was not evident in gait evolution using HyperNEAT in hardware \cite{yos:clune}. 
HyperNEAT took full advantage of the regularity of the problem and produced gaits more regular, coordinated, and effective than all other gaits evolved for the QuadraTot platform in the past.
This is evident in the servo position plots where the gaits produced in this study have smooth and repeating curves while gaits produced by HyperNEAT in hardware produced much more ragged and random curves.
The difference in regularity between the two studies is most likely due to the number of evaluations that were done in gait evolution. Additionally, the noise in the real world was extremely high, making learning in hardware difficult, which may have further impeded the search for higher-performing, regular gaits.

While this study was able to produce the highest fitness gait that has ever been tried on the QuadraTot, it is worth noting that most of the gaits in simulation did not transfer well to reality. 
Many gaits that performed very well in simulation performed poorly on the real robot because of the physical limitations and differences between simulation and real-world, such as limited servo power, frictional differences, and other minor differences in the robot physique, all of which accumulates to have a large impact in performance.
The most oft-seen problem with real trials, regardless of the actions taken to reduce it, was the servo shutdowns. 
Because we limited gait frequencies, HyperNEAT was forced to turn to other means for maximizing distance. 
So it pushed the servo positions to their limits by extending and retracting from one extreme to the other at a pace which was difficult for the servos to handle. 
Another reason for the shutdowns was that the servos could not easily support the robot's weight well, which often led to the servos shutting down after a few seconds of walking off-ground. 

These problems led to difficulties in repeatability. 
The robot is weak and the power of the motors changes between and even within trials. 
And since the gaits evolved in this study stressed the servos a lot, after one successful run, there was no guarantee that the robot would be able to perform that same gait succesfully next time. 
In that sense, the gaits produced in this study is perhaps worse than the gaits produced by RL PoWER Spline \cite{haocheng} or by genetic algorithm with a physical simulator \cite{glette} whose gaits were repeatable and less taxing on the robot. 


% table!
\begin{table}
\caption{The number of evaluations per run (gen x orgs), simulated and real trial velocities in [cm/s]. Data used from Yosinksi et al. \cite{yos:clune}, Glette et al. \cite{glette}, and Shen et al. \cite{haocheng}.
*17.8 cm/s was recorded on Glette et al. \cite{glette} but the same gait tried on our robot averaged over ten runs was 12.951 cm/s}  %\tablabel{results}
\begin{center}
\begin{tabular}{|l|c|c|c|c|}
\hline
                                         & Num. Evaluations  & Sim. Velocity  & Real. Velocity \\
\hline
Best parameter optimization learning    &153    & --    & 4.64 \\
\hline
Hardware HyperNEAT                  & 180         & --         &   9.7     \\
\hline
Simulated GA              & 60000       & 16.4       &   12.951*     \\
\hline
RL PoWER Spline                          & 300         & --         &   11.05 \\
\hline
Simulated HyperNEAT                      & 40000       & 25.4       &   14.5 \\
\hline
\end{tabular}
\end{center}
\end{table}

%
%

\begin{figure}
\begin{center}
\vspace{1cm}
\epsfig{file=fit_gen.eps, width=8cm}
\caption[ ]{Fitnesses of the champion gaits averaged over 20 runs, measured in distance travelled over 12 seconds [cm/s]. These gaits outperformed gaits evolved in simulation using Genetic Algorithm by 54.3\% in simulation and 5.4\% better in reality while outperforming gaits evolved using HyperNEAT in hardware by 49.5\%.}
\end{center}
\end{figure}


\begin{figure}
\begin{center}
\vspace{1cm}
\epsfig{file=freq_fitness.eps, width=8cm}
\caption[ ]{Average frequencies of the gaits averaged over 20 runs. After implementing a policy which punished gaits with high frequencies, HyperNEAT was able to find high-performing, low frequency gaits. This policy was implemented to reduce servo shutdowns which heavily affected the reality performance of the robot.}
\end{center}
\end{figure}


\begin{figure}
\begin{center}
\vspace{1cm}
\epsfig{file=servo_pos.eps, width=8cm}
\caption[ ]{Top: A plot of servo positions for a gait produced in this study. Bottom: A plot of servo positions for a gait produced by HyperNEAT in hardware. Figure taken from Yosinski et al. \cite{yos:clune}. The smoothness, symmetry, and regularities of the top plot shows that by using a simulator, HyperNEAT was able to fully exploit the geometry of the problem.}
\end{center}
\end{figure}


% Section 6
\section{Conclusion and Future Work}

We have presented a new reinforcement-learning-based algorithm for
optimizing a quadrupedal gait for linear speed. We implemented and
tested six learning strategies for parameterized gaits and compared
them to gaits produced by neural networks evolved with the HyperNEAT
generative encoding. Though both methods resulted in an improvement
over the robot’s previous \naive gaits, based on the statistics
collected, RL PoWER has a more elegant and consistent performance.

Though over 900 trials have been made to investigate the applicability
of RL PoWER to quadruped robots. It is difficult to gather the enough
trials that would be necessary to properly rank the methods. One
direction for future work could be to obtain many more trials. But due
to the physical limitations, obtaining One solution to this is
simulation. Because of the low cost of simulation, it would produce
the necessary volume of trials to allow the learning methods to be
effective, and the hardware trials would serve to continuously ground
and refine the simulator. \cite{glette2012evolution-of-locomotion-in-a-simulated}

One guess led by this study is that for feedback oriented tasks,
reinforcement learning methods are more fit in natural. Despite the
complexities of HyperNEAT, a simpler algorithm on the code level
delivered better performance in general.  Also, evolvable spline
interpolation is shown to be simple and representationally powerful at
the same time. Evolvable splines can serve as a general representation for various other
learning problems.

%\section{Future work}
\seclabel{futureWork}

Because all trials were done in hardware, it was difficult to gather
the many trials that would be necessary to properly rank the methods statistically.
One direction for future work could be to obtain many more trials. However, a more effective extension might be to combine frequent trials in
simulation with infrequent trials in hardware~\cite{bongard}.  The simulation would produce the
necessary volume of trials to allow the learning methods to be
effective, and the hardware trials would serve to continuously ground
and refine the simulator.  One could also guide evolution to the most
fertile territory by penalizing gaits that produced large
discrepancies between simulation and reality~\cite{koos2010crossing}. Another natural extension would be to allow gaits that sensed the position of the robot and other variables to enable the robot to adjust to its physical state, instead of providing an open-loop sequence of motor commands. All of these approaches would likely improve the quality of automatically generated gaits for legged robots, which will hasten the day that humanity can benefit from their vast potential. 



% Section 7
\section{Acknowledgments}

This work was supported in part by NSF CDI Grant ECCS 0941561, NSF
Creative-IT grant 0757478, and NSF Postdoctoral Research Fellowship
DBI-1003220.

%The content of this paper is solely the responsibility of the authors
%and does not necessarily represent the official views of the
%sponsoring organizations.''

% List any people not on the team who helped you with your project: Your
% TA, other people you consulted with or had useful discussions (say in 
% a word or two what they did), people who proofread your report, and 
% any external code you used (libraries etc).
%\begin{itemize}
%\item Hod Lipson, Cornell Computational Synthesis Lab: adviser.
%\item Jim T\o rreson, University of Oslo: adviser.
%\item Juan Zagal, University of Chile: designed and printed robot and provided code for hand tuned gait.
%\item Jeff Clune, CCSL Red Couch: collaborated on HyperNEAT implementation/testing.
%\item Cooper Bills, Cornell University: assisted with Wii tracker development.
%\item Anshumali Srivastava, Cornell University: Teaching Assistant.
%\end{itemize}



%%%%%%%%%%%%%%%%%%%%%%

\footnotesize
\bibliography{references.bib}
\bibliographystyle{apalike}

\end{document}
