\section{Summary of Results and Discussion}
\seclabel{results}

In the interest of space, we present only a brief summary of gait
learning results in \tabref{results}.  All results were obtained using
the QuadraTot robot, with learning performed using either physical or
simulated trials.  These promising results suggest that evolutionary and
learning algorithms may enable more rapid
exploration of the space of robotic gaits and other behaviors than
would be possible manually.



%%%%%%% NEW TABLE in cm/s
\begin{table*}
\begin{center}
\begin{tabular}{|r|c|c|c|c|c|}
\hline
Learning Algorithm                       &   & Type of  &         &           & Single \\
(number of trials)                       & Paper  & Trials   & Average & Std. Dev. & Best Gait \\
\hline                                        
\hline                                        
Hand-coded gait                          & \cite{yosinski2011evolving-robot-gaits}  & --       & 1.09   &   --   &       \\
\hline                                        
Policy Gradient Descent (174)            & \textquotedbl  & physical & 1.34   &   1.56 &       \\
\hline                                                         
Uniform Random Hill Climbing (234)       & \textquotedbl  & physical & 1.66   &   0.97 &       \\
\hline                                                          
Random search (200)                      & \textquotedbl  & physical & 1.99   &   1.45 &       \\
\hline                                                          
Gaussian Random Hill Climbing (284)      & \textquotedbl  & physical & 2.12   &   1.27 &       \\
\hline                                                          
Nelder-Mead simplex (172)                & \textquotedbl  & physical & 2.61   &   0.71 &       \\
\hline                                                          
Linear Regression (153)                  & \textquotedbl  & physical & 2.97   &  2.73  &       \\
\hline                                                          
HyperNEAT (540)                          & \textquotedbl  & physical & 5.56   &   1.35 &  9.68 \\
\hline                                                      
RL PoWER (900) & \cite{shen2012learning-fast-quadruped}        & physical   & 7.62   & 2.1   &  11.05 \\
\hline                                                      
Alternate Genetic Algorithm (60,000) & \cite{Glette2012Evolution} & simulated & (14.2)   &  (5.2) &  (17.8) 13.0$^*$ \\
\hline                                                      
HyperNEAT (800,000) & \cite{lee2012INPREP-evolving-quadruped-gaits} & simulated &    &    &  14.5 \\
\hline
\end{tabular}
\caption{The average and standard deviation of the best gait speeds,
  in cm/sec, found using various encodings and learning
  algorithms. The single best observed gait is also included where
  available. All reported speeds are those observed in the physical
  world; i.e. for simulated trials we report only the speed of the
  gait when transferred to reality, not the speed observed in
  simulation.  Speeds in parentheses are reported in
  \cite{Glette2012Evolution} and were observed on a QuadraTot
  configured similarly, though not identically, to
  that in \cite{yosinski2011evolving-robot-gaits}. The speed with an asterisk
  is the measurement of the same gait on the QuadraTot from
  \cite{yosinski2011evolving-robot-gaits}.}  \tablabel{results}
\end{center}
\end{table*}




\subsection{Open Robotics}

Despite this promise, the field remains small, partly because robots
are expensive and difficult to modify. Access to cheap, customizable
robots could increase the number of researchers able to participate in
the field. Moreover, in nearly all of the papers mentioned previously,
the robots were custom-made, preventing teams at other universities
from reproducing the results of other groups and or testing new
algorithms on a robotic platform used in a previous study. That, in
turn, slows the progress of science because it is difficult to
interpret whether the variance in results between different studies
was due to the algorithms used or the robotic platform those
algorithms were tested on.

Some robot platforms are emerging, but they tend to be wheeled robots
without complex kinematics, such as the ePuck~\citep{mondada2009puck}.
For this reason we have introduced several low-cost, open-source, and
easily customizable robotic platforms, such as the QuadraTot
\citep{yosinski2011evolving-robot-gaits} and Aracna
\citep{lohmann2012aracna-an-open-source-quadruped}. These robots can
be 3D printed from design files available online, allowing other
researchers to customize the body's design.  These robots will enable
multiple users to compare data across a small number of lightweight,
low-cost evolutionary robotic platforms.
