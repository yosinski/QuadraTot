% !TEX root =main.tex
%\figgp{quadratotWhiteBooties}{.45}{quadratot_simulator}{.45}{captioncaption}

We performed the experiments on the QuadraTot quadrupedal robot platform~\cite{yos:clune}.
It has 9 degrees-of-freedom (DOF): two joints per leg and one joint that allowed rotation along the robot's midline. %we def want a picture of the cool robot. I propose putting the picture of the robot on the left and the picture of the simulated robot on the right of the same figure. That also does a nice job of highlighting the theme of the paper (robot+sim) in one figure. Try to get this figure to show up as early as possible in the paper (e.g top of the 2nd page). 
QuadraTot has open source software and hardware\footnote{A list of all of the parts and all software, including the simulator, is available for free at http://quadratot.yosinski.com.} %JMC: This should really be a Creative Machines/Cornell web address. Can you ask Jason if he agrees and, if so, take the lead on making sure it gets made?
, including many parts that can be 3D printed. These traits enable other labs to perform research with it, providing a platform for comparing different gait-learning algorithms. 
To date there are results on the platform from nine different learning algorithms from three previous publications \cite{yos:clune,glette,haocheng}. We identified the robot's position using infrared LEDs on the robot which was tracked using a Wiimote~\cite{yos:clune}.

Both in this work and that of Glette et al.~\cite{glette}, the joints are powered by Robotis Dynamixel servos; five AX-18A servos for the inner joint of each leg and the single midline joint, and four AX-12A servos for the the outer joint of each leg, which requires less power and can thus have less expensive motors. The servos were sent new positions at 40Hz via the Pydynamixel library. 

Each servo has a built-in safety mechanism that shuts itself off to prevent damage if the servo's current, range, temperature, or torque was too high \cite{robotis}. During evolution, this safety mechanism frequently activated, and did so inconsistently, adding significant noise to the evaluation process. As pointed out in Yosinski et al. \cite{yos:clune}, gaits generated during evolution are highly variable and generate many shutdowns because they forced the servos to exert too much torque or command positions that are out of range. %JMC: Does requesting a value out of range really cause the servo to shut down? That seems like something we could easily fix (by never allowing an out-of-range value). Now that we do not allow values out of range, we still see lots of shutdowns...making me think that the problem was really torque. What do you think?
To minimize shutdowns, we limited the allowable range of movement for the inner, outer, and hip joints to [-85\degree, +60\degree], [-113\degree, +39\degree], and [-28\degree, +28\degree], respectively. We also implemented the Smart Cropping System from Shen et al. \cite{haocheng}, which prevents combinations of joint positions for the inner and outer joint of each leg that generate extreme amounts of torque. A final method of reducing torque was to reduce the weight of the robot. Yosinski et al.~\cite{yos:clune} had the small Linux computer that performed all computation on the robot, but we removed it and sent commands to the robot via a cable. 




% % figure of the quadratot in its natural stance
% \begin{figure}
% \begin{center}
% \vspace{1cm}
% % just using a photo from Haocheng's paper! Should replace if needed be           
% \epsfig{file=quadratot_fig.eps, width=8cm}
% \caption[ ]{The QuadraTot robotic platform. It is open-sourced and 3-D printable, which has allowed for other researchers to use the robot for their research around the world. The robot has 9 DOF, 2 per leg and a hip joint which connects the two symmetrical halves of the robot. The robot uses Dynamixel AX-18 and AX-12 servos for actuation. Figure taken from Shen et al. \cite{haocheng}.}
% \label{fig:quadratot}
% \end{center}
% \end{figure}
% %

%\figp{sixLegPositions}{.6}{Allowed and disallowed leg position angles. \edit{which are which? see JMC comment too} %JMC: Isn't this described in Haocheng's paper? I don't think we need to show this here, because it is not a new thing in our paper and is not central to our story. If we do keep it, instead of 3 sentences of text saying what the columns are, just put the headings ``Allowed'', ``End of Allowable Range'', and ``Disallowed''
%The figure shows an example of leg positions allowed and not allowed on the robot. The two left positions are allowed. The middle two positions are on the borderline. The right two positions are considered as ``fierce'' and are disallowed. Instead, these 
%Disallowed gaits are cropped to the middle two positions to prevent servo shutdowns. Figure taken from Shen et al. \cite{haocheng}}


% \begin{figure}
% \begin{center}
% \vspace{1cm}
% \epsfig{file=gaitspace.eps, width=8cm}
% \caption[ ]{A map of possible leg positions. Manually placed position cropping systems limited possible leg positions to avoid servo shutdowns and the robot hitting itself. The sum of the inner and outer servo positions cannot drop below 730, the smart cropping border, marked by the red line. Figure taken from Shen et al. \cite{haocheng}.}
% \end{center}
% \end{figure}
%\figp{smartCrop}{.6}{A map of possible leg positions. Manually placed position cropping systems limited possible leg positions to avoid servo shutdowns and the robot hitting itself. The sum of the inner and outer servo positions cannot drop below 730, the smart cropping border, marked by the red line. Figure taken from Shen et al. \cite{haocheng}.}


% \begin{figure}
% \begin{center}
% \vspace{1cm}
% \epsfig{file=wiimote.eps, width=8cm}
% \caption[ ]{The Wiimote used to track the motion of the robot. The Wiimote tracked an IR LED placed on top of the robot and sent the location data to a nearby computer which recorded the position of the robot with respect to time. Figure taken from Yosinski et al. \cite{yos:clune}.}
% \end{center}
% \end{figure}
\figgp{wiiMote_crop.jpg}{.6}{robot_led_crop.jpg}{.6}{The Wiimote used to track the motion of ther robot. The Wiimote was used to track the IR LED placed on top of the robot and send the location data to a nearby computer which recorded the position of ther robot with respect to time. Figure taken from Yosinski et al. \cite{yos:clune}.}

















