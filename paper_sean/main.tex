% First draft
% last modified
% by Suchan Sean Lee
% on 7/29/12

\documentclass{llncs}
\usepackage{graphicx}
\usepackage{ownstyles}
\graphicspath{{../figures/}}
%
\newcommand{\degree}{\ensuremath{^\circ}}
%
%
\begin{document}

%% BEGINNING of TITLE Section
%
%\title{Evolving Quadruped Gaits for Physical Robots with a Bio-Inspired Generative Encoding Using a Simulator}
\title{Evolving robot gaits for physical robots with generative encodings: the benefits of simulation.}

\author{Suchan Lee \and Jason Yosinski \and Jeff Clune \and Hod Lipson}
\institute{Cornell University, 239 Upson Hall, Ithaca, NY 14853, USA \\ 
\email{sl746@cornell.edu, }
\email{yosinski@cs.cornell.edu, }
\email{jeffclune@cornell.edu}}
\maketitle
%
%% END of TITLE section
%
%
%% BEGINNING of ABSTRACT section
%
\begin{abstract}
Applications of walking robots often call for the ability to walk as
quickly, efficiently, or with as little power as possible.  Gaits to
achieve these objectives may designed manually or learned by repeated
trial and error.
Learning approaches differ in their starting assumptions, some
tweaking the parameters of a hand-tuned model, others exploring a
reasonably compact parameter space, and still others beginning with
few assumptions besides periodicity.

This study compares the performance of two methods of learning gaits:
local search of parameterized motion models and evolution of
artificial neural networks using the HyperNEAT encoding.

We tested six different learning strategies for parameterized gaits,
including uniform and Gaussian random hill climbing, policy gradient
reinforcement learning, Nelder-Mead simplex, and a new method that
uses linear regression to build a model of the fitness landscape and
predict promising areas of parameter space for further exploration.
While all parameter search methods outperformed a manually designed
gait, only the Nelder-Mead simplex and linear regression strategies
beat a random baseline strategy.

The HyperNEAT gaits performed considerably better than all
parameterized local search methods.  Successful evolved gaits showed
complex motion patterns containing multiple fundamental frequencies,
but they also demonstrated reuse of patterns among several motors.
Both served to produce quick gaits.

All tests were performed directly in hardware on a quadruped robot
with nine degrees of freedom.  To the authors' knowledge this is the
first time HyperNEAT gaits have been evolved or tested in hardware.

\end{abstract}
%
%% END of ABSTRACT section
%
%%BEGINNING of  paper section... replace the inputs with the actual texts later to make one, big LaTeX file

\section{Introduction}
\section{Introduction}

What the problem is...

\figp{aracna_black_2}{.75}{Example figure...}


%
%
\section{Experimental Setup}
\subsection{Platform}
\subsubsection{Robot Hardware.}
% !TEX root =main.tex
The robot platform used for this project was a quadrupedal robot named the QuadraTot\footnote{Robot plans and part files for 3D printing are freely available at http://quadratot.yosinski.com} designed in the Cornell Creative Machines Lab by Juan Zagal~\cite{yos:clune}.
It has 9 degrees-of-freedom (DOF), with 2 DOF for the inner joint and the outer joint of each leg and 1 DOF for the hip joint of the robot. 
The QuadraTot is an open-sourced and 3-D printable robot, which allows it to be used in other labs for research, providing a platform with which people could compare the results of different gait learning algorithms. 
Currently, there are published data and results from 9 different learning algorithms on the platform from INSERT different publications. %JMC fill in INSERT, and dump all cites here. This is a good place to cite haochengs paper, which is fun for him and Jason, but does not distract from our story.


The body of the robot was printed using an Objet Connex 500 3-D Printing System. 
It was actuated using 9 Robotis Dynamixel servos: 5 AX-18A servos and 4 AX-12A servos. 
The stronger AX-18A servos were used to actuate the inner joints of the robot and the hip joint while the AX-12A servos were used to actuate the outer joints. 
The servos were attached to the robot using off-the-shelf components and parts.
The servos have a range of motion of [0, 1023], roughly equivalent to a physical range of [-12\degree, +120\degree]. 

The Dynamixel AX-12A and AX-18A servos have a safety feature called \emph{Alarm Shutdown} that would turn off the servo if the servo's current, range, temperature, or torque was too high \cite{robotis}. 
The ones most relevant to this study were shutdowns due to high torque and range. 
As has been reported previously with QuadraTot\cite{yos:clune}, many of the gaits that were tested on the QuadraTot forced the servos to exert too much torque or would command out of range positions and cause the servos to shutdown. In order to address this problem and avoid collisions with its own body, a motion cropping system was implemented on the QuadraTot. 
In addition to placing limits on the range of motions for the inner joints, outer joints, and the hip joint of [-85\degree, +60\degree], [-113\degree, +39\degree], and [-28\degree, +28\degree] respectively, an extra cropping system named the smart cropping system was adopted from Shen et al. \cite{haocheng} to further prevent servo shutdowns from leg positions where the combination of the two angles interact to produce an extreme amount of torque. This smart cropping algorithm calculates a leg position by summing the servo position values from the inner and outer joints of a leg, \emph{\textbf{g}} (\emph{\textbf{g}} = $j_{inner}$ + $j_{outer}$) %JMC - do they sum the position values, or the joint angles? Since all the previous extremes are expressed in angles, the writing makes me expect angles...
and modifying the servo positions from fierce gaits, which usually have relatively low \emph{\textbf{g}}s, to remap them onto the boundary of a quasi-trangular area in a 2-D space defined by $j_{inner}$ and $j_{outer}$. %the term fierce sounds silly and non-professional (sorry!) I took out the place where you defined it, and we only use it in two other places. No need to invent a definition only to use it 2-3 times...can you swap in different, more descriptive words?
Through experimentation, we found that a \emph{\textbf{g}} of 730 provides a good balance between performance and safety. %JMC - This description is not great. The last sentence's use of g doesn't make sense with the equation...is g a threshold or the sum? Please try to clean up this description. 


The robot was controlled by CompuLab Fit-PC2, a compact computer running Ubuntu Linux 10.10. 
In a previous study \cite{yos:clune}, this computer was placed on the robot to keep the robot untethered, but for this study the computer was taken off the robot to reduce weight, because the weight of the robot seemed to be one of the factors contributing to servo shutdowns.
The servos were sent positions from the computer at 40Hz via the Pydynamixel library. 


We identified the robot's position via infrared LEDs on the robot that could be tracked by a Nintendo Wii remote.
%JMC you sure there was more than one IR LED? I thought it was just one IR and one visible for us to tell if there is power to the LEDs. If it is just one, switch to singular instead of plural in the previous sentence
To maximize its viewable window, the Wii remote was fixed to the ceiling pointing down. It communicated with the robot computer via bluetooth using the CWiid library~\cite{yos:clune}. 
%JMC The paragraph about the Wii mote was very wordy. I think I trimmed it by half. You should try to be similarly compact the rest of the paper. Compare your version to mine and try to see how I condensed things. 

% % figures
% % figure of the quadratot in its natural stance
% \begin{figure}
% \begin{center}
% \vspace{1cm}
% % just using a photo from Haocheng's paper! Should replace if needed be           
% \epsfig{file=quadratot_fig.eps, width=8cm}
% \caption[ ]{The QuadraTot robotic platform. It is open-sourced and 3-D printable, which has allowed for other researchers to use the robot for their research around the world. The robot has 9 DOF, 2 per leg and a hip joint which connects the two symmetrical halves of the robot. The robot uses Dynamixel AX-18 and AX-12 servos for actuation. Figure taken from Shen et al. \cite{haocheng}.}
% \label{fig:quadratot}
% \end{center}
% \end{figure}
% %
\figp{robot_whitebg}{.6}{The QuadraTot robotic platform, on which numerous gait evolution papers are based. Our method of evolving gaits in simulation with the HyperNEAT generative encoding and transferring them to the robot produced faster gaits than all previously published techniques. 
%It is open-sourced and 3-D printable, which has allowed for other  researchers to use the robot for their research around the  world. 
%JMC The previous sentence (which I commented out) is alright for the main text, but figures are very expensive real estate. The goal of our paper is not to sell this robot, but to sell the work we did. 
There are 9 degrees of freedom: 2 per leg, and a single `hip joint' that connects  the two symmetrical halves of the robot.
%JMC: If you use which instead of that, you need a comma before which. For example, this is wrong: There are 9 degrees of freedom: 2 per leg, and a single `hip joint' which connects  the two symmetrical halves of the robot. This is right. There are 9 degrees of freedom: 2 per leg, and a single `hip joint' that connects  the two symmetrical halves of the robot. This is also right: There are 9 degrees of freedom: 2 per leg, and a single `hip joint,' which connects  the two symmetrical halves of the robot.
% The robot uses Dynamixel  AX-18 and AX-12 servos for actuation. %JMC We said this already in methods
% Figure taken from Shen et  al. \cite{haocheng}.%JMC Actually it's from yos:clune, but I think it is ok to leave this out, since this is just a stock photo of the QuadraTot.
 }
%JMC Let's put this figure on the first page (before the introduction), unless you can get a series of pictures that show a gait. (This might be hard to do). A new trend in papers is to have a sexy image on the first page, and Hod will want to see this. I also think it makes people want to read the paper more. 

% \begin{figure}
% \begin{center}
% \vspace{1cm}
% \epsfig{file=legpos.eps, width=8cm}
% \caption[ ]{The figure shows an example of leg positions allowed and not allowed on the robot. The twoleft positions are allowed. The middle two positions are on the borderline. The right two positions are considered as ``fierce'' and are disallowed. Instead, these gaits are cropped to the middle two positions during gait performance by the cropping mechanism placed to prevent servo shutdowns. Figure taken from Shen et al. \cite{haocheng}}
% \end{center}
% \end{figure}
\figp{sixLegPositions}{.6}{The figure shows an example of leg positions allowed and not allowed on the robot. The twoleft positions are allowed. The middle two positions are on the borderline. The right two positions are considered as ``fierce'' and are disallowed. Instead, these gaits are cropped to the middle two positions during gait performance by the cropping mechanism placed to prevent servo shutdowns. Figure taken from Shen et al. \cite{haocheng}}


% \begin{figure}
% \begin{center}
% \vspace{1cm}
% \epsfig{file=gaitspace.eps, width=8cm}
% \caption[ ]{A map of possible leg positions. Manually placed position cropping systems limited possible leg positions to avoid servo shutdowns and the robot hitting itself. The sum of the inner and outer servo positions cannot drop below 730, the smart cropping border, marked by the red line. Figure taken from Shen et al. \cite{haocheng}.}
% \end{center}
% \end{figure}
\figp{smartCrop}{.6}{A map of possible leg positions. Manually placed position cropping systems limited possible leg positions to avoid servo shutdowns and the robot hitting itself. The sum of the inner and outer servo positions cannot drop below 730, the smart cropping border, marked by the red line. Figure taken from Shen et al. \cite{haocheng}.}


% \begin{figure}
% \begin{center}
% \vspace{1cm}
% \epsfig{file=wiimote.eps, width=8cm}
% \caption[ ]{The Wiimote used to track the motion of ther robot. The Wiimote was used to track the IR LED placed on top of the robot and send the location data to a nearby computer which recorded the position of ther robot with respect to time. Figure taken from Yosinski et al. \cite{yos:clune}.}
% \end{center}
% \end{figure}
\figgp{wiiMote_crop.jpg}{.6}{robot_led_crop.jpg}{.6}{The Wiimote used to track the motion of ther robot. The Wiimote was used to track the IR LED placed on top of the robot and send the location data to a nearby computer which recorded the position of ther robot with respect to time. Figure taken from Yosinski et al. \cite{yos:clune}.}

\subsubsection{Simulator.}
% !TEX root =main.tex
The simulator\footnote{Available for Linux and Windows at http://quadratot.yosinski.com} was the same one used in Glette et al. \cite{glette}. C++ code for the the Nvidia PhysX physics engine approximately describes the QuadraTot, including the mass and size of the robot components and its degrees of freedom. In the simulator, each individual joint range was limit as described above, but Smart Cropping was not included because it hindered performance by limiting the types of gaits evolution could explore in early generations. 

\figp{simulator_outline}{.6}{Physical representation of the QuadraTot in simulation. It captures the important parts of the robot, such as the number and types of the joints, mass, and the rough shape and lengths. The simulator was written by Kyrre Glette in C++ using NVIDIA PhysX software library and visualized using OpenGL. Figure taken from Glette et al. \cite{glette}.}


\subsubsection{HyperNEAT.}
% !TEX root =main.tex
<<<<<<< HEAD
HyperNEAT is an algorithm for evolving artificial neural networks (ANNs)~\cite{stanley3}. 
=======
HyperNEAT is a neuroevolution method that evolves artificial neuron networks (ANNs) using evolutionary algorithms.
HyperNEAT has shown to evolve large neural networks that represent many features present in brains such as repetitions and regularities \cite{stanley2002evolving}. 
This encoding has previously been applied for gait-learning \cite{yos:clune} and for evolving 2-D and 3-D objects \cite{clune:lipson20113d} and has been successful in displaying complex, natural characteristics such as symmetries and repetition of themes often found in nature.

>>>>>>> d94bbc7cf7652452770aa6b2b525d2448624000f

HyperNEAT indirectly encodes ANNs using compositional pattern producing networks (CPPNs) which represent genomes encoding ANN phenotypes \cite{stanley2009hypercube}, \cite{stanley2007CPPN}. 
The motivation for CPPNs is that complex patterns can be produced by determining the attributes of their phenotypic components as a function of their geometric location. 
This is based on the belief that in nature, cells in a body differentiates and specializes in different things with respect to their location. 
In real life, a cell in natural organisms cannot determine its location in space without the help of chemical gradients but \emph{in silico}, cells can be given their geometric coordinates, which CPPNs exploit. 
%A cell in natural organisms cannot determine its location in space by itself and requires chemical gradients are used to signal its location \cite{carroll}. 
%In contrast, \emph{in silico}, cells can be given their geometric coordinates and CPPNs use this to their advantage. 


Each CPPN is a directed graph in which every node is a mathematical function, such as sine or Gaussian. 
Moreover, a CPPN genome allows functions to be made of other functions, allowing coordinate frames to be combined and hierarchies to develop, allowing for several different themes to appear such as a repeating theme (sine) with symmetry (Gaussian). 
%For example, a sine function could develop a repeating theme which is passed onto a Gaussian function to create a repeating series of symmetrical motifs. 
These functions allow evolution to exploit various properties that allow HyperNEAT to find regularities without further human intervention \cite{stanley2009hypercube,clune2009evolving}. %such as symmetry (e.g. Gaussian) and repetition (e.g. sine function) \cite{stanley1}. 
%Because evolution is able to exploit such properties, things like gaits are able to feature repetition and symmetry on its own without human intervention \cite{clune1}. 


There are weighted links between nodes in CPPNs that are used to multiply the signal in each link and determine the magnitude of effect from each node. 
In HyperNEAT, a CPPN genome takes in Cartesian coordinates (X,Y) of the source and the target nodes, and a constant bias value and outputs the weight between the input and hidden layer and a weight between the hidden layer and output (assuming that there is a hidden layer, as in the case for this study). 
All pairwise combinations of source and target nodes are passed into a CPPN to determine the weight of each ANN link. 


HyperNEAT is first neuroevolutionary algorithm that has shown capabilities in exploiting the geometry of the problem \cite{stanley2009hypercube}, \cite{clune2009sensitivity}.
This is possible because the connection weights are a function of geometric positions of these nodes. 
If positions inputed to CPPNs represent aspects of the problem relevant to the solution, HyperNEAT could use that information to its advantage.  

Previous work has shown that Hyper
HyperNEAT has shown to evolve large neural networks that represent many features present in brains such as repetitions and regularities . 
This encoding has previously been applied for gait-learning \cite{yos:clune} and for evolving 2-D and 3-D objects \cite{clune:lipson} and has been successful in displaying complex, natural characteristics such as symmetries and repetition of themes often found in nature.


HyperNEAT evolves the CPPNs according to the principles of the Neuroevolution of Augmenting Topologies (NEAT) algorithm \cite{stanley2006exploiting}. 
The NEAT algorithm has three major components \cite{stanley2006exploiting}. 
First, it starts with small genomes that encode simple networks and complexifies them via mutations and add nodes and links to the network which allows the alogithm to evolve the network topology and weight. 
Second, NEAT has a fitness-sharing system that preserves diversity in the system and allows for new innovations to be tuned by evolution before competing them against more adapted rivals. 
Third, the historical information stored in genes helps to perform crossover in a way that is effective, yet avoids the need for expensive topological analysis. A full explanation of NEAT can be found in Stanley and Miikkulainen \cite{stanley2006exploiting}. 


In this study, the ANN configuration from evolving gaits with HyperNEAT in hardware \cite{yos:clune} was used. 
In this ANN configuration, the ANN has a fixed topology that consists of three 3 X 4 Cartesian grids of nodes for the input, hidden, and output layers. 
The inputs to the substrate were the angles requested in the prevous time step for each of the 9 joints of the robot and a sine and cosine wave to facilitate periodic motion. 
The outputs of the substrate at each time step were nine numbers (for each joint) in the range [-1, 1] which were scaled to the range [0, 1023], the allowable ranges for the servos. 
As in Yosinski et al. \cite{yos:clune}, we requested four times as many commanded positions from HyperNEAT ANNs and averaged over four commands at a time in order to reduce the possibility of switching from extreme negative to extreme positive numbers at a very high frequency. 

%%% FIGURES %%%

% \begin{figure}
% \begin{center}
% \vspace{1cm}
% \epsfig{file=CPPN.eps, width=8cm}
% \caption [ ]{Compositional Pattern Producing Networks (CPPNs). CPPN  genomes use mathematical functions to generate regularities such as symmetry and repetitions. CPPN genomes allow functions to be made of other functions, allowing multiple regularity motifs to be present. Figure taken from Stanley \cite{stanley2}.}
% \end{center}
% \end{figure}
\figp{hyperneat_bug_example}{.6}{Compositional Pattern Producing Networks (CPPNs). CPPN  genomes use mathematical functions to generate regularities such as symmetry and repetitions. CPPN genomes allow functions to be made of other functions, allowing multiple regularity motifs to be present. Figure taken from Stanley \cite{stanley2007CPPN}.}

% \begin{figure}
% \begin{center}
% \vspace{1cm}
% \epsfig{file=hyperneat.eps, width=8cm}
% \caption [ ]{Producing ANNs from CPPNs. In HyperNEAT, weights are specified as a function of the Cartesian coordinates for source and input nodes and a constant bias for each connection between the source and input. All pairwise combinations of source and target node coordinates are iteratively passed into CPPNs to determine the weight of each ANN link. Figure taken from Clune et al. \cite{clune2}}.
% \end{center}
% \end{figure}
\figp{hyperneatExplanation}{.6}{Producing ANNs from CPPNs. In HyperNEAT, weights are specified as a function of the Cartesian coordinates for source and input nodes and a constant bias for each connection between the source and input. All pairwise combinations of source and target node coordinates are iteratively passed into CPPNs to determine the weight of each ANN link. Figure taken from Clune et al. \cite{clune2011performance}}.



\subsection{Methods}
\section{Methods}

how we tried to solve it...

%
%
\section{Results and Discussion}
HyperNEAT, used with a physical simulator, produced gaits that outperformed all previous gaits tried and recorded on the QuadraTot. 
The fastest gait evolved by HyperNEAT in simulation and transfered onto the real robot recorded a speed of 14.5 cm/s, faster than the previously fastest gait published in Glette et al. \cite{glette}. 
In their paper, Kyrre et al. \cite{glette} recorded 17.8 cm/s in their robot in Norway \cite{glette}, but to make sure that there were no differences in robots, the same gait was also tested on the robot used for our experimental testing.
The same gait, ran by using the gait file sent to us by Kyrre Glette, recorded an average speed of 12.951 cm/s and a maximum of 13.76 cm/s on the same QuadraTot platform in the Cornell Creative Machines Lab where Yosinski et al. \cite{yos:clune} and this study was done.
HyperNEAT, used with a simulator, outperformed Kyrre et al's \cite{glette} gait by 5.4\% with much fewer evaluations performed per run, at 40000 evaluations per run with 200 organisms and 200 generations as opposed to Kyrre et al's \cite{glette} 60000 evaluations per run with 200 organisms and 300 generations.
The best gait evolved using HyperNEAT with simulator performed 49.5\% better than the best gait evolved using HyperNEAT in hardware \cite{yos:clune}. 
This is most likely due to the difference in the number of evaluations between the two studies. 
The previous study, HyperNEAT in hardware, only had 180 evaluations while in this study, a simulator allowed a much higher number -- 20000 -- of evaluations. 
Another minor reason for this may be that the implementation placed to reduce gait frequency may have helped in producing gaits that were less susceptible to shutdowns, as Yosinski et al. \cite{yos:clune} did not have such things.
Although most gaits started off its evolution process by producing fast and high frequency gaits, with more generations, the gaits dramatically reduced its frequencies while continually increasing their fitness. The average frequency of the final gaits was 1.29 Hz per servo. 

Gaits produced in this study confirmed that with the use of a physical simulator, HyperNEAT is able to fully exploit the geometry of the problem without human intervention, which was not evident in gait evolution using HyperNEAT in hardware \cite{yos:clune}. 
HyperNEAT took full advantage of the regularity of the problem and produced gaits more regular, coordinated, and effective than all other gaits evolved for the QuadraTot platform in the past.
This is evident in the servo position plots where the gaits produced in this study have smooth and repeating curves while gaits produced by HyperNEAT in hardware produced much more ragged and random curves.
The difference in regularity between the two studies is most likely due to the number of evaluations that were done in gait evolution. Additionally, the noise in the real world was extremely high, making learning in hardware difficult, which may have further impeded the search for higher-performing, regular gaits.

While this study was able to produce the highest fitness gait that has ever been tried on the QuadraTot, it is worth noting that most of the gaits in simulation did not transfer well to reality. 
Many gaits that performed very well in simulation performed poorly on the real robot because of the physical limitations and differences between simulation and real-world, such as limited servo power, frictional differences, and other minor differences in the robot physique, all of which accumulates to have a large impact in performance.
The most oft-seen problem with real trials, regardless of the actions taken to reduce it, was the servo shutdowns. 
Because we limited gait frequencies, HyperNEAT was forced to turn to other means for maximizing distance. 
So it pushed the servo positions to their limits by extending and retracting from one extreme to the other at a pace which was difficult for the servos to handle. 
Another reason for the shutdowns was that the servos could not easily support the robot's weight well, which often led to the servos shutting down after a few seconds of walking off-ground. 

These problems led to difficulties in repeatability. 
The robot is weak and the power of the motors changes between and even within trials. 
And since the gaits evolved in this study stressed the servos a lot, after one successful run, there was no guarantee that the robot would be able to perform that same gait succesfully next time. 
In that sense, the gaits produced in this study is perhaps worse than the gaits produced by RL PoWER Spline \cite{haocheng} or by genetic algorithm with a physical simulator \cite{glette} whose gaits were repeatable and less taxing on the robot. 


% table!
\begin{table}
\caption{The number of evaluations per run (gen x orgs), simulated and real trial velocities in [cm/s]. Data used from Yosinksi et al. \cite{yos:clune}, Glette et al. \cite{glette}, and Shen et al. \cite{haocheng}.
*17.8 cm/s was recorded on Glette et al. \cite{glette} but the same gait tried on our robot averaged over ten runs was 12.951 cm/s}  %\tablabel{results}
\begin{center}
\begin{tabular}{|l|c|c|c|c|}
\hline
                                         & Num. Evaluations  & Sim. Velocity  & Real. Velocity \\
\hline
Best parameter optimization learning    &153    & --    & 4.64 \\
\hline
Hardware HyperNEAT                  & 180         & --         &   9.7     \\
\hline
Simulated GA              & 60000       & 16.4       &   12.951*     \\
\hline
RL PoWER Spline                          & 300         & --         &   11.05 \\
\hline
Simulated HyperNEAT                      & 40000       & 25.4       &   14.5 \\
\hline
\end{tabular}
\end{center}
\end{table}

%
%

\begin{figure}
\begin{center}
\vspace{1cm}
\epsfig{file=fit_gen.eps, width=8cm}
\caption[ ]{Fitnesses of the champion gaits averaged over 20 runs, measured in distance travelled over 12 seconds [cm/s]. These gaits outperformed gaits evolved in simulation using Genetic Algorithm by 54.3\% in simulation and 5.4\% better in reality while outperforming gaits evolved using HyperNEAT in hardware by 49.5\%.}
\end{center}
\end{figure}


\begin{figure}
\begin{center}
\vspace{1cm}
\epsfig{file=freq_fitness.eps, width=8cm}
\caption[ ]{Average frequencies of the gaits averaged over 20 runs. After implementing a policy which punished gaits with high frequencies, HyperNEAT was able to find high-performing, low frequency gaits. This policy was implemented to reduce servo shutdowns which heavily affected the reality performance of the robot.}
\end{center}
\end{figure}


\begin{figure}
\begin{center}
\vspace{1cm}
\epsfig{file=servo_pos.eps, width=8cm}
\caption[ ]{Top: A plot of servo positions for a gait produced in this study. Bottom: A plot of servo positions for a gait produced by HyperNEAT in hardware. Figure taken from Yosinski et al. \cite{yos:clune}. The smoothness, symmetry, and regularities of the top plot shows that by using a simulator, HyperNEAT was able to fully exploit the geometry of the problem.}
\end{center}
\end{figure}

%
%
\section{Conclusion and Future Work}
\section{Conclusion and Future Work}

We have presented a new reinforcement-learning-based algorithm for
optimizing a quadrupedal gait for linear speed. We implemented and
tested six learning strategies for parameterized gaits and compared
them to gaits produced by neural networks evolved with the HyperNEAT
generative encoding. Though both methods resulted in an improvement
over the robot’s previous \naive gaits, based on the statistics
collected, RL PoWER has a more elegant and consistent performance.

Though over 900 trials have been made to investigate the applicability
of RL PoWER to quadruped robots. It is difficult to gather the enough
trials that would be necessary to properly rank the methods. One
direction for future work could be to obtain many more trials. But due
to the physical limitations, obtaining One solution to this is
simulation. Because of the low cost of simulation, it would produce
the necessary volume of trials to allow the learning methods to be
effective, and the hardware trials would serve to continuously ground
and refine the simulator. \cite{glette2012evolution-of-locomotion-in-a-simulated}

One guess led by this study is that for feedback oriented tasks,
reinforcement learning methods are more fit in natural. Despite the
complexities of HyperNEAT, a simpler algorithm on the code level
delivered better performance in general.  Also, evolvable spline
interpolation is shown to be simple and representationally powerful at
the same time. Evolvable splines can serve as a general representation for various other
learning problems.

%
%
%% BEGINNING of BIBLIOGRAPHY

\editbox{Make this a proper bibliography. See main.tex and references.bib in paper\_aracna}
\begin{thebibliography}{1}
\bibitem {bongard:lipson}
Bongard, J., Zykov, V., and Lipson, H.:
Resilient Machines Through Continuous Self-Modeling.
Science. 314(5802):1118-1121 (2006)
%
\bibitem {valsalam:mii}
Valsalam, V. and Miikkulainen, R.:
Modular Neuroevolution for Multilegged Locomotion.
In: Proceedings of the Genetic and Evolutionary Computation Conference, pp. 265--272 (2008)
%
\bibitem {kohl:stone}
Kohl, N. and Stone, P.:
Machine Learning for Fast Quadrupedal Locomotion.
In: The Ninetheenth National Conference on Artificial Intelligence (AAAI 2004), pp. 611--616. San Francisco (2004)
%
\bibitem {clune1}
Clune, J., Beckmann, B., Ofria, C., and Pennock, R.:
Evolving Coordinated Quadruped Gaits with the HyperNEAT Generative Encoding.
In: Proceedings of the IEEE Congress on Evolutionary Computation, pp. 2764--2771 (2009)
%
\bibitem {clune3}
Clune, J., Ofria, C., and Pennock, R.:
The Sensitivity of HyperNEAT to Different Geometric Representations of a Problem.
In: Proceedings of the Genetic and Evolutionary Commuptation Conference, pp. 675--682 (2009)
%
\bibitem {stanley1}
Stanley, K. O., D'Ambrosio, D. B., and Gauci, J.:
A Hypercube-Based Encoding for Evolving Large-Scale Neural Networks.
Artificial Life. 15(2), 185-212 (2009)
%
\bibitem {clune2}
Clune, J., Stanley, K. O., Pennock, R., and Ofria, C.:
On the Performance of Indirect Encoding Across the Continuum of Regularity.
IEEE Transactions on Evolutionary Computation. 15, 346-367 (2011)
%
\bibitem {yos:clune}
Yosinski, J., Clune, J., Hidalgo, D., Nguyen, S., Zagal, J. C.:
Evolving Robot Gaits in Hardware: the Hyperneat Generative Encoding vs. Parameter Optimization.
In: Proceedings of the 20th European Conference on Artificial Life, pp. 11--18 (2011)
%
\bibitem {glette}
Glette, K., Kalus, G., Zagal, J. C., and Torresen, J.:
Evolution of Locomotion in a Simulated Quadruped Robot and Transferral to Reality.
In: Proceedings of the Seventeenth International Symposium on Aritifical Life and Robotics, (2012)
%
\bibitem {haocheng}
Shen, H., Yosinski, J., Kormushev, P., Caldwell, D. G., Lipson, H.:
Learning Fast Quadruped Robot Gaits with the RL PoWER Spline Parameterization.
In: AIMSA 2012 Workshop on Advances in Robot Learning and Human-Robot Interaction, (2012)
%
\bibitem {robotis}
Robotis:
User's Manual, Dynamixel AX-12.
URL: http://www.trossenrobotics.com/images/productdownloads/AX-12(English).pdf
%
\bibitem {clune:lipson}
Clune, J., Lipson, H.:
Evolving Three-Dimensional Objects with a Generative Encoding Inspired by Developmental Biology.
In: Proceedings of the European Conference on Artificial Life, pp. 144--148 (2011)
%
\bibitem {stanley2}
Stanley, K. O.:
Compositional Pattern Producing Networks: A Novel Abstraction of Development.
Genetic Programming and Evolvable Matter. 8(2), 131--152 (2007)
%
\bibitem {stanley3}
Stanley, K. O.:
Exploiting Regularity Without Development.
In: Proceedings of the AAAI Fall Symposium on Developmental Systems, (2006)
%
\bibitem {stanley4}
Stanley, K. O., Miikkulainen, R.:
Evolving Neural Networks Through Augmenting Topologies.
Evolutionary Computation, 10(2):99-127 (2002)
%
\bibitem {jakobi}
Jakobi, N., Husbands, P., and Harvey, I.:
Noise and the Reality Gap: The Use of Simulation in Evolutionary Robotics.
In: Advances in Artificial Life: Proceedigns of the 3rd European Conference on Artificial Life, pp. 704--720, Springer-Verlang, (1995)
%%
\bibitem {koos1}
Koos, S., Mouret, J.-B., Doncieux, S.:
Crossing the Reality Gap in Evolutionary Robotics by Promoting Transferable Controllers.
In: Proceedings of the 12th Annual Conference on Genetic and Evoutionary Computation, pp. 119--126 (2010)
%
\bibitem {carroll}
Carroll, S.:
Endless Forms Most Beautiful: The New Science of Evo Devo and the Making of the Animal Kingdom.
Norton, New York (2005)
%
\bibitem {bongard}
Bongard, J.:
Synthesizing Physically-Realistic Environmental Models from Robot Explorations.
In: Proceedings of the 9th European Confrence on Advances in Artificial Life, pp. 806--815 (2007)
%
\bibitem {zagal}
Zagal, J., Ruiz-del Solar, J., and Vallejos, P.:
Back to Reality: Crossing the Reality Gap in Evolutionary Robotics.
In: Proceedings of IAV 2004, the 5th IFAC Symposium on Intelligent Autonomous Vehicles, (2004)
%
\bibitem {koos2}
Koos, S., Mouret, J.-B., Doncieux, S.:
The Transferability Approach: Crossing the Reality Gap in Evolutionary Robotics.
IEEE Transactions on Evolutionary Computation, (2011)

\end{thebibliography}
%
%
\end{document}
