\section{Introduction and Background}

%\acmFig{robot_close.jpg}{.8}{The quadruped robot. The translucent parts
%  were created with 3D-printing technology. Videos of the gaits
%  evolved for this robot can be viewed at http://bit.ly/geccogait}

%%%%%%% NEW TABLE
\begin{table}
\begin{center}
\begin{tabular}{|r|c|c|c||c|}
\hline
                                         & Average & Std. Dev. \\
\hline                                    
\hline                                    
Previous hand-coded gait                 & 5.16   &   --     \\
\hline
Random search                            & 9.40   &   6.83   \\
\hline
Uniform Random Hill Climbing             & 7.83   &   4.56   \\
\hline
Gaussian Random Hill Climbing            & 10.03  &   6.00   \\
\hline
Policy Gradient Descent                  & 6.32   &   7.39   \\
\hline
Nelder-Mead simplex                      & 12.32  &   3.35   \\
\hline
Linear Regression                        & 14.01  &  12.88   \\
\hline
Evolved Neural Network              &        &          \\
(HyperNEAT)                              & 29.26  &   6.37   \\
\hline
\end{tabular}
\caption{The average and standard deviation of the best gaits found
  for each algorithm during each of three runs, in body
  lengths/minute.  Videos of the gaits evolved for this robot can be
  viewed at http://bit.ly/geccogait} \tablabel{results}
\end{center}
\end{table}



%Gaits for walking robots are often designed with some explicit or
%implicit goal in mind.  For some applications, the design criteria may
%be obvious --- perhaps the robot needs to move as quickly or as
%efficiently as possible --- but other times the objective is more complicated,
%requiring simultaneous optimization of several desired traits, each
%with its own relative importance.  The different combinations of desired traits and the relative weight placed on each can produce drastically different gaits.    For example, Honda's Asimo
%\cite{chestnutt2006footstep} and Boston Dynamic's Big Dog \cite{raibert2008bigdog} both require
%gaits that are relatively quick, power efficient, and robust to
%changing terrain, but they vary widely in the importance placed
%on each attribute.  Big Dog's gait is optimized for much more difficult terrain than Asimo's, resulting in a gait of a completely different form.

%Once the design goals are decided upon, gaits may be obtained by one

%Legged robots have the potential to access many types of terrain
%unsuitable for wheeled robots, but doing so requires the creation of a
%gait specifying how the robot is to walk.  Gaits may be designed
%by one of several methods: either manually by an expert or
%via computer learning algorithms.  Learned gaits
%offer several advantages over manually designed gaits.  Automatically
%learning gaits can save valuable engineering time and can allow
%customization of gaits to a particular robot and its unique actuators.
%Most importantly, in many cases learned gaits can outperform manually
%designed gaits~\cite{hornby2005autonomous, valsalam2008modular}.



%Previous work has shown that quadruped gaits perform better when they
%are \emph{regular} (i.e.\ when the legs are
%coordinated)~\cite{clune2009evolving,
%  clune2011performance,valsalam2008modular}. For example, HyperNEAT
%produced fast, natural gaits in part because its bias towards regular
%gaits created coordinated movements that outperformed gaits evolved by
%an encoding not biased towards regularity~\cite{clune2009evolving,
%  clune2011performance}. One of the motivations of this paper is to
%investigate whether any learning method biased towards regularity
%would perform well at producing quadruped gaits, or whether
%HyperNEAT's high performance is due to additional factors, such as its
%abstraction of biological development (described below). We test this
%hypothesis by comparing HyperNEAT to six local search algorithms with
%a parametrization biased toward regularity.

%A major motivation of this paper is to simply evolve effective gaits
%for a physical robot. Because HyperNEAT performed well in simulation,
%it is interesting to test whether it can produce a fast gait for a
%physical robot. It is additionally interesting to test how more
%traditional gait optimization techniques compete with evolutionary
%algorithms when evolving in hardware.
%


% Put back in for reference... some good text here I think

% Applications of walking robots often call for the ability to walk as 
% quickly, efficiently, or with as little power as possible.  Often 
% these optimizations are done manually by an expert who designs and 
% tweaks a gait specifically for a given objective.  Other groups have 
% used learning methods to generate gaits optimized for some metric. 
% Approaches differ in their starting assumptions, some essentially 
% tweaking the parameters of a hand-tuned model \cite{chernova}, others 
% exploring a reasonably compact parameter space \cite{kohl}, and still 
% others beginning with few assumptions besides periodicity 
% \cite{zykov}. 
%  
% We aimed to strike a middle ground between these approaches.  Our 
% motion generator did not rely on hand-tweaked gaits, but it did use 
% parameterized gaits which, by their nature, impose some assumptions on 
% the answers produced.  We then used machine learning to design gaits 
% for a quadruped robot with these models.  This paper presents a 
% comparison of the different learning methods implemented.  Most 
% methods created walks that are several times faster than the original 
% hand-tuned gait.  We invite readers with short attention spans to view 
% a video of some of our results online here: 
%  
% 
% %\url{http://www.youtube.com/watch?v=ODoiOj9DdGg} 
% \texttt{http://www.youtube.com/watch?v=ODoiOj9DdGg} 

