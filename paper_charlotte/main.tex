\documentclass[letterpaper]{article}
\usepackage{natbib,alife13}

\title{QuadraRex: An Open-Source Quadruped Platform for Evolutionary Robotics}
\author{Sara Lohmann, Eric Gold, Jeremy Blum, Jason Yosinski \and Hod Lipson \\
\mbox{}\\
Cornell University, 239 Upson Hall, Ithaca, NY 14853 \\
\texttt{sml253@cornell.edu}}



\begin{document}
\maketitle

\begin{abstract}
We report on a new open-source quadruped robotic platform, QuadraRex, for evolutionary robotics experiments. The robot comprises four legs with two joints each, for a total of eight kinematic degrees of freedom. Four-bar linkage mechanisms in each leg drive the
pitch of the knee joint and the hip joint remotely, allowing the motors to remain in the robot center, thereby reducing the inertia of each leg. Due to these non-conventional kinematics, the robot requires non intuitive movements for locomotion that are challenging for learning. We suggest that this platform can provide a low-cost, accessible platform for testing and comparing physical evolutionary robotics algorithms.
\end{abstract}



\section{Introduction}

We address the need for a low-cost platform with non-intuitive walking kinematics to explore evolutionary robotics. QuadraRex is the third evolutionary learning quadruped robot developed by the Creative Machines Lab \citep{HL, JY}. A common feature among the quadruped robots is two actuators that drive the flexion/extension of a knee and hip joint around parallel axes in each leg. The original Creative Machines Lab quadruped robot favored starfish-like movements \citep{HL}. The second quadruped robot, QuadraTot \citep{JY}, developed spider-like movements, but was limited by its weight. When creating the QuadraRex, we designed the hardware to compliment fast spider-like movements.

\begin{figure}[t]
\begin{center}
\includegraphics[width=.45\textwidth]{fig1.jpg}
\caption{QuadraRex: an open-sourced quadruped robot platform. All instructions and downloads are publicly available at http://creativemachines.cornell.edu/quadrarex \citep{WEB}.}
\label{fig1}
\end{center}
\end{figure}



\section{Hardware}

\begin{figure}[t]
\begin{center}
\includegraphics[width=.23\textwidth]{fig3.pdf}
\includegraphics[width=.23\textwidth]{fig4.pdf}
\caption{Crank-rocker 4 bar linkage controlling flexion/extension of the knee and hip joints.}
\label{fig3}
\end{center}
\end{figure}

%\begin{figure}[t]
%\begin{center}
%\includegraphics[width=2.25in,angle=0]{fig4.pdf}
%\caption{Crank-rocker 4 bar linkage controlling flexion/extension of the hip joint.}
%\label{fig4}
%\end{center}
%\end{figure}

The hardware of QuadraRex evolved from the previous Creative Machines Lab quadruped robots. We kept the same two degree of freedom pitch joint scheme but decreased the weight and constrained the movement of the joints to create faster spider-like movement. To prevent starfish-like movement, the legs were constrained. We designed the robot with two four bar mechanisms to drive the joints in each leg. With a four bar mechanism in place, the leg moves at a fraction of the output angle of the actuator. Figure~\ref{fig3} shows the crank-rocker system, where the input crank (link OA) is actuated by a servo, the rocker is the leg (link CB), and the fixed link is OC. In this configuration, we keep the servo motors contained in the center of the robot, thereby reducing the inertia of each leg. The four bar mechanisms satisfied the design goal of making a robot that had non-traditional movements. We avoided designing a robot similar to an off-the-shelf robot to create something unique. To make the robot accessible as an open-source platform, it was designed to be 3D printed. The 3D print files and CAD files can be accessed by the public to be modified and printed.
\\
We designed the robot to be lightweight to focus on the goal of evolutionary robotics. We use a single LiPo 11.1V battery to power 8 Dynamixel AX-18 servo motors and an ArbotiX microcontroller. The main processing occurs on an external computer, which allows the robot to have cheaper and lighter components on-board. We use the space in the center of the body to house the battery to reduce material weight and cost. QuadraRex weighs less than 1 kg.
\begin{figure}[t]
\begin{center}
\includegraphics[width=2.25in,angle=0]{fig5.jpg}
\caption{Renderd CAD model.}
\label{fi52}
\end{center}
\end{figure}

\section{Software}
The software is open-sourced code written in Python based on the evolutionary learning algorithms of the QuadraTot \citep{JY}. We provide the controller code on our website \citep{WEB}. We use an external computer for processing. We use an infrared light emitting diode on the robot along with an external Wii remote to provide feedback of the distance traveled. The external computer receives feedback from the Wii remote and internal servo sensors and then processes the information to send the next command to position the servos. The communication between the external computer and the ArbotiX on-board microcontroller occurs over wireless XBee.

\section{Specifications}
\begin{figure}[t]
\begin{center}
\includegraphics[width=2.25in,angle=0]{fig2.jpg}
\caption{Robot printed as one piece with support material intact.}
\label{fig2}
\end{center}
\end{figure}
We 3D printed the robot as one part (Figure~\ref{fig2}). The robot was completed after approximately 26 hours, and used approximately 1000 g of model material and 1500 g of support material. We estimated the cost of the printed parts to be \$410 by using the estimated model material cost to be \$1/4.5g and the support material cost to be \$1/8g. This cost estimate is highly variable depending on the type of material used. The robot has no removable printed parts except for the battery cover. We will make the robot available to print as separate parts for smaller 3D printer tray sizes. Printing the robot as separate parts will reduce the amount of support material and cost to print. We estimated the total cost to be less than \$1500 for the custom robot platform. We made these estimates based on the material calculations above and prices from trossenrobotics.com.

\begin{table}[h]
\center{
\begin{tabular}{|c|c|}
\hline
Part & Cost\\
\hline\hline
3D Print Materials & \$410.00\\
ArbotiX Robocontroller Kit & \$189.00\\
Dynamixel AX-18A Robot Actuator (x8) & \$721.28\\
3S 11.1V 2000mAh Pro Lite LiPo Battery & \$72.99\\
LiPo Battery Balance Charger Kit & \$69.99\\
Cables, Connectors, Misc & \$27.78\\
\hline\hline
\bf Total & \bf \$1491.04\\
 \hline
\end{tabular}
}
\vskip 0.25cm
\caption{Estimated total cost. A specifiic parts list is on our website \citep{WEB}.}
\end{table}



\section{Acknowledgements}

This work was supported by
%NSF Creative-IT grant 0757478.
the National Science Foundation's Office of Emerging Frontiers in Research and Innovation (grant number 0735953).


\footnotesize
\bibliographystyle{apalike}
\bibliography{QuadraRex}
\end{document}

