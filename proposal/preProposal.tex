\section{Team}

1.The Team. Names of the people working on the project. We suggest
working in groups of about three, but can be convinced otherwise if
you have a good reason. If you can’t find a group, email the TAs and
we will assign you to a group.



\section{Problem statement}

2.Problem statement and motivation. A statement of the problem, issue,
or task that you’re interested in studying, and why it is interesting
or important.



\section{I/O specification}

3.I/O Specification. A clear, concise description of what the final
system will do in terms of I/O behavior: What will it take in, and
what it will produce.



\section{Background reading (optional)}

4.Background Reading (OPTIONAL in pre-proposal stage). A list of
relevant readings that you’ll use to gain some background in your
selected topic.



\section{General approach (optional)}

5.General Approach (OPTIONAL in pre-proposal stage). A high-level
description of the general approach you’ll use (e.g. heuristic search,
learning, rules, belief networks). This section should have a
subsection entitled “Where’s the AI” where you explicitly articulate
the AI component of this project.



\section{System architecture and work plan (optional)}

6.System Architecture and work plan (OPTIONAL in pre-proposal
stage). Explain the main components of the system, how they can be
independently developed and independently tested, and who will do
what.



\section{Data sources (optional)}

7.Data sources (OPTIONAL in pre-proposal stage). If your project
involves analyzing real data (e.g. stock market data, sport data, user
behavior), identify potential sources for this data.



\section{Evaluation plans (optional)}

8.Evaluation Plans (OPTIONAL in pre-proposal stage). An explicit,
coherent plan for quantitatively and/or qualitatively evaluating the
system. In particular, identify (a) metrics that will help you measure
performance, e.g. time or space to solve a problem, quality of a
solution, or competition or comparison against a standard approach (b)
a simple “toy” problem that you can use for early testing of your
system – something that should be trivial to solve, (b) a “hard” test
case, that will serve as you ultimate test.



\section{Schedule}

9.Schedule. A schedule of work indicating the dates by which you plan
to complete components of the system. This should be presented as a
table listing dates and milestones.

